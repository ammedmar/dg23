\documentclass[12pt]{amsart}
\usepackage{microtype}
\usepackage{amssymb}
\usepackage{mathtools}
\usepackage{tikz-cd}
\usepackage{mathbbol} % changes \mathbb{} and adds more support
\usepackage{csquotes}
\usepackage{enumitem}
\usepackage{setspace} % For line spacing
\usepackage{mathptmx} % Times New Roman for both text and math
\usepackage[scaled=.92]{helvet} % Helvetica, scaled to match Times New Roman
\usepackage{layout}

% set up 12pt Times New Roman
\renewcommand{\rmdefault}{ptm}
\renewcommand{\sfdefault}{phv}
\renewcommand{\ttdefault}{pcr}

%\usepackage{layout}
\pagestyle{empty}

\setlength{\parindent}{0pt}
\usepackage[letterpaper,
	margin=1.87cm,
	hoffset=0cm,
	voffset=0cm,
	headheight=0cm,
	headsep=0cm,
	marginparwidth=0cm,
	marginparsep=0cm,
	footskip=0cm]{geometry} % Sets paper size and margins

% line spacing
%\onehalfspacing % Alternatively, you can use \singlespacing to try to meet the "no more than six lines per inch" requirement

% bibliography
\usepackage[
	backend=biber,
	style=numeric, % alphabetic, numeric, remove sorting=none
	sorting=nyt,
	backref=false,
	url=false,
	doi=false,
	isbn=false,
	eprint=false]{biblatex}

\setlength{\bibitemsep}{0.0cm}
\renewbibmacro{in:}{}  % don't display "in:" before the journal name
\AtEveryBibitem{\clearfield{pages}}  % don't show page numbers
\input{aux/usualcmds}
\addbibresource{../usualpapers.bib}
\addbibresource{../bibliography.bib}
% !TEX root = ../proposal.tex

\newcommand{\coho}{\rH^\bullet}
\renewcommand{\cochains}{\rC^\bullet}
\newcommand{\KH}{\mathrm{KH}_\bullet}
\newcommand{\giottoTDA}{\raisebox{-.5pt}{\includegraphics[scale=.08]{aux/giotto}}}
\DeclareMathOperator{\ad}{ad}
\DeclareMathOperator{\barconst}{\mathbf{B}}
\newcommand{\desus}[1]{s^{\mbox{\scriptsize -#1}}}
\DeclareMathSymbol{\antishrik}{\mathord}{operators}{"3C}
\newcommand{\shrik}{!}
\newcommand{\cdga}{\mathsf{cdga}}
\newcommand{\cdgl}{\mathsf{cdgl}}
\newcommand{\com}{\mathcal{C}om}
\newcommand{\lie}{\mathcal{L}ie}
\newcommand{\Der}{\mathrm{Der}}
\newcommand{\fullline}{\par\noindent\rule{\textwidth}{0.4pt}}


\usepackage{csquotes}
\usepackage{enumitem}
\usepackage[letterpaper, margin=1.87cm]{geometry} % Sets paper size and margins
\setlength{\parindent}{0pt}
\usepackage{setspace} % For line spacing
\usepackage{mathptmx} % Times New Roman for both text and math
\usepackage[scaled=.92]{helvet} % Helvetica, scaled to match Times New Roman

% set up 12pt Times New Roman
\renewcommand{\rmdefault}{ptm}
\renewcommand{\sfdefault}{phv}
\renewcommand{\ttdefault}{pcr}

% line spacing
%\onehalfspacing % Alternatively, you can use \singlespacing to try to meet the "no more than six lines per inch" requirement

%%%%%%%%%%%%%%%%%%%%%%%%%%%%%%%%%%%%%5

\title[Effective algebro-homotopical constructions]{Effective algebro-homotopical constructions \\ with applications to data science and quantum field theory}

\author{Anibal~M.~Medina-Mardones}
%\address{A.M-M., Max Planck Institute for Mathematics \and University of Notre Dame}
%\email{\href{mailto:ammedmar@mpim-bonn.mpg.de}{ammedmar@mpim-bonn.mpg.de}}

\begin{document}
	\vspace*{-0pt}
	\maketitle
	\thispagestyle{empty}
	% !TEX root = ../proposal.tex

\textsc{\large Effective Algebro-Homotopical Constructions with Applications to Data Science and Quantum Field Theory}
--- \hfill \textsc{\large Anibal M. Medina-Mardones}

\bigskip\textsc{Introduction}.
There is a tense trade-off in algebraic topology having roots reaching back to the beginning of its modern form.
This tension can be illustrated with the concept of cohomology.
The first approaches, dating back to Poincar\'e, are based on the subdivision of a space into simple pieces.
These elementary shapes are made to generate a free graded module and their spatial relations define the differential used to compute cohomology.
This definition makes certain geometric properties of cohomology, for example excision, fairly clear.
Yet, it is not easy to show that a continuous map of spaces induces a map between their associated cohomologies.
The functoriality just alluded to is trivial when defining cohomology in terms of homotopy classes of maps to Eilenberg--MacLane spaces, but the passage to the homotopy category erases geometric and combinatorial information and the resulting definition is not well suited for concretely presented spaces.
The tension this example illustrates manifests itself in many other important contexts, and the trade-off between concreteness and functoriality is as central today as it was almost a century ago.

\smallskip\textsc{Objectives}.
\textit{My long-term vision is to ease this tension by developing concrete constructions of homotopical invariants currently defined only indirectly; and, by doing so, extend the reach of algebraic topology into fields beyond pure mathematics, including Data Science and Quantum Field Theory.}

\smallskip We will organize the presentation of this program around four interrelated objectives.

\smallskip
(1) Compute primary \& secondary cohomology operations of triangulated spaces. \hfill (MSc1-2, PhD1-2)\par
(2) Compute Steenrod operations on Khovanov homology of knots and links. \hfill (MSc3, PhD3)\par
(3) Compute Steenrod barcodes and persistent cup-lengths of real-world datasets. \hfill (MSc4-5, PhD4)\par
(4) Build a bridge between state sum and functorial topological quantum field theories. \hfill (MSc6, PostDoc)\par

%\smallskip
%Each of these tangible contributions will follow from novel understanding of the algebro-homotopical properties of topological spaces, with a strong emphasis on computation and effectiveness.
	% !TEX root = ../proposal.tex

\smallskip\textsc{Background}.
As mentioned in the introduction, Poincaré's definition of the cohomology of a space $X$ relies on its cochains, denoted simply as $\cochains$, which arise from a cellular decomposition of $X$, typically a triangulation.
This construction is local in nature, allowing for a piecewise analysis of $X$, and is thus well-suited for concrete computations.
The computer algebra system (CAS) \texttt{SageMath}, among others, is a tool where this approach to cohomology is implemented.
Unfortunately, cohomology—or equivalently, the quasi-isomorphism type of $\cochains$—omits crucial homotopical information.
As an illustrative example, consider the spaces $\bC P^2$ and $S^2 \vee S^4$, the union of a 2- and a 4-sphere over a point.
Despite their cohomology groups being isomorphic, these spaces are not homotopy equivalent.
To distinguish them, one can employ the Alexander--Whitney product in $\cochains$ to reveal that their ring structures in cohomology are not isomorphic \cite{alexander1936ring, whitney1938products}.
This product-enhanced version of $\cochains$ retains the original's computability and is also supported in \texttt{SageMath}.
Considering the suspension of these spaces leads to two non-homotopic spaces $\Sigma(\mathbb{C} P^2)$ and $\Sigma(S^2 \vee S^4)$ whose cohomologies are isomorphic as graded rings.
To distinguish them we can consider Steenrod squares on $\rH^\bullet$ which are effectively computable through the cup-$i$ products \cite{steenrod1947products}, a structure on $\cochains$ also implemented in \texttt{SageMath}, which generalizes the Alexander-Whitney structure.
These enhancements on $\cochains$ would not suffice to distinguish a similar comparison involving a quaternionic projective space.
In that case, we can use odd prime power operations, but these are not available on mainstream computer algebra systems yet.

A ground-breaking result in modern homotopy theory, due to Mandell \cite{mandell2001padic,mandell2006homotopy_type}, asserts that under certain finiteness assumptions, all the homotopical information of $X$ is captured by an enhancement of the cup-$i$ product structure on $\cochains$.
This complete invariant is known as an $E_\infty$-algebra and serves as the conceptual cornerstone of our program.
The author and others provided a concrete local description of this structure on $\cochains$ \cite{medina2020prop1, mcclure2003multivariable, berger2004combinatorial}, which opened the door for the development of effective constructions of progressively richer homotopical invariants.
Consequently, it enabled their potential integration into computer algebra systems, as well as into more concrete fields such as knot theory, toric geometry, and lattice quantum field theory.

%%As mentioned in the introduction, Poincar\'e's definition of the cohomology of a space $X$ relies on the cochains $\cochains$ of a cellular decomposition of $X$.
%%This construction is local and well-suited for concrete computations.
%%Unfortunately, cohomology, or equivalently the quasi-isomorphism type of $\cochains$, forgets much homotopical information.
%%For example, the cohomology groups of $\mathbb{C} P^2$ and $S^2 \vee S^4$, the union of a $2$- and a $4$-sphere over a point, are isomorphic despite these spaces not being homotopy equivalent.
%%By choosing triangulations for these spaces, the Alexander--Whitney product can be used to show that the ring structure in cohomology distinguishes them.
%%An algorithm for this computation is available for example on \texttt{SageMath}.
%
%
%A fundamental step in the direction of our goal is the description of a ring structure on $\rH^\bullet$ using the Alexander--Whitney product on simplicial cochains $\cochains$ \cite{alexander1936ring, whitney1938products}.
%Let us illustrate with an example the additional information captured by this ring structure.
%The cohomology groups of $\mathbb{C} P^2$ and $S^2 \vee S^4$, the union of a $2$- and a $4$-sphere over a point, are isomorphic despite these spaces not being homotopy equivalent.
%By choosing triangulations for these spaces, the Alexander--Whitney product can be used to show that the ring structure in cohomology distinguishes them.
%An algorithm for this type of computation is available for example on \texttt{SAGE}.
%Considering the suspension of these spaces leads to two non-homotopic spaces $\Sigma(\mathbb{C} P^2)$ and $\Sigma(S^2 \vee S^4)$ whose cohomologies are isomorphic as graded rings.
%Distinguishing these effectively lead us to a novel step in the direction of our goal.
%Let us consider a prime $p$.
%Steenrod made $\rH^\bullet$ it into a module over his algebra $\cA_p$ \cite{steenrod1962cohomology}.
%This structure is explicitly induced from the $E_\infty$-algebra structure on simplicial cochains by means of the cup-$(p,i)$ products.
%These generalize to odd primes the cup-$i$ products of Steenrod, and were introduced and implemented, using effective versions of May's operadic methods \cite{may1970general}, in \cite{medina2021may_st,medina2021comch}.
%The action of $\cA_2$ suffices to distinguish $\Sigma(\mathbb{C} P^2)$ and $\Sigma(S^2 \vee S^4)$, but and odd prime will be needed for similar comparisons using the suspension of quaternionic projective spaces, for example.
%
%
%A ground-breaking result in modern homotopy theory, due to Mandell \cite{mandell2006homotopy_type}, is that, under certain finiteness assumptions, all the homotopical information of $X$ is captured by a so-called $E_\infty$-algebra structure on $\cochains$.
%This complete algebro-homotopical invariant is the conceptual cornerstone of our program, since, after work by the author and others, it can be given a concrete local description (\cite{mcclure2003multivariable,berger2004combinatorial,medina2020prop1}).
%This allows for the effective description of progressively richer homotopical invariants that can be implemented into computer algebra systems (CAS) such as \texttt{ComCH}, \texttt{SageMath}, and \texttt{Maple}.
	% !TEX root = ../proposal.tex

\section*{(1) Effective Algebro-Homotopical Constructions and Their Implementations \\ (PhD1, MSc1, MSc2)}

A foundational goal of the systematic study of topological spaces up homotopy equivalence is the development of algebraic invariants, such as cohomology.
%Our program focuses on spaces that can be decomposed into cells.
As mentioned in the introduction, Poincar\'e's definition of cohomology $\rH^\bullet$ relies on the cochains $\cochains$ of a cellular space.
A ground-breaking result in modern homotopy theory, due to Mandell \cite{mandell2006homotopy_type}, is that, under certain finiteness assumptions, all the homotopical information of a cellular space is captured by a so-called $E_\infty$-algebra structure on $\cochains$.
Aided by this language, the first goal of our program can be more precisely stated as follows:

\smallskip\noindent(1a) Effectively describing progressively richer layers of homotopical data using the $E_\infty$-algebra on $\cochains$.\par
\noindent(1b) Implementing these constructions into computer algebra systems (CAS) such as \texttt{ComCH}, \texttt{SAGE}, and \texttt{Maple}.

\smallskip\noindent These developments will render the invariants within these layers accessible through computer-based computations, which is indispensable for their use in data science (Section 3), and, additionally, they will play an essential role in providing local descriptions of topological action functionals on triangulated space-times (Section 4).

\medskip\noindent\textbf{Earlier work}.
A fundamental step in the direction of our goal is the description of a ring structure on $\rH^\bullet$ using the Alexander--Whitney product on simplicial cochains $\cochains$ \cite{alexander1936ring, whitney1938products}.
Let us illustrate with an example the additional information captured by this ring structure.
The cohomology groups of $\mathbb{C} P^2$ and $S^2 \vee S^4$, the union of a $2$- and a $4$-sphere over a point, are isomorphic despite these spaces not being homotopy equivalent.
By choosing triangulations for these spaces, the Alexander--Whitney product can be used to show that the ring structure in cohomology distinguishes them.
An algorithm for this type of computation is available for example on \texttt{SAGE}.
Considering the suspension of these spaces leads to two non-homotopic spaces $\Sigma(\mathbb{C} P^2)$ and $\Sigma(S^2 \vee S^4)$ whose cohomologies are isomorphic as graded rings.
Distinguishing these effectively lead us to a novel step in the direction of our goal.
Let us consider a prime $p$.
Steenrod made $\rH^\bullet$ it into a module over his algebra $\cA_p$ \cite{steenrod1962cohomology}.
This structure is explicitly induced from the $E_\infty$-algebra structure on simplicial cochains by means of the cup-$(p,i)$ products.
These generalize to odd primes the cup-$i$ products of Steenrod, and were introduced and implemented, using effective versions of May's operadic methods \cite{may1970general}, in \cite{medina2021may_st,medina2021comch}.
The action of $\cA_2$ suffices to distinguish $\Sigma(\mathbb{C} P^2)$ and $\Sigma(S^2 \vee S^4)$, but and odd prime will be needed for similar comparisons using the suspension of quaternionic projective spaces.

%Steenrod operations in the mod $p$ cohomology of spaces, which together with the Bockstein homomorphism provide a complete account of the mod $p$ cohomology functor.

\medskip\noindent\textbf{{\sc Relations}}.
The next layer of homotopical structure after Steenrod operations comes from the relations these operations satisfy.
By group homology computations, or more abstractly Mandell's Theorem, we known that there must be some structure on the $E_\infty$-algebra on $\cochains$ inducing these so-called \textit{Cartan} and \textit{Adem relations}.
%We referred to these as Cartan and Adem coboundaries.
Recently, these structures have been effectively described over any prime for Cartan's, and over the even prime for Adem's \cite{medina2020cartan,medina2023oddcartan,medina2021adem}.

\medskip\noindent\textbf{Goals}.
My team will culminate this research direction and consolidate it into useful software.
More specifically, we will complete the following tasks:

\smallskip\noindent(MSc1) Implementing the existing Cartan and Adem structures over the even prime.\par
\noindent(MSc2) Implementing the existing Cartan structure over odd primes.\par
\noindent(PhD1) Discovering and implementing the Adem structure over odd primes.

\smallskip\noindent As we will detail in Section 4, these structures have already proven important in the classification of topological phases of matter.
Additionally, making them accessible through computer computations represents the initial stride toward integrating secondary operations into persistent cohomology, as discussed in Section 3.

\smallskip\noindent\textbf{Methodology}.
We have already established a strategic partnership with the Ontario Research Centre for Computer Algebra (ORCCA), and the software development aspects of this project will be executed in close collaboration with this institution.

\medskip\noindent\textbf{{\sc Khovanov homology} (PhD2)}
$\mathrm{KH_\bullet}$ is a powerful algebraic invariant of knots and links which refines the Jones polynomial \cite{khovanov2000khovanov}.
It is effectively computable from a chain complex associated with a knot diagram, and several implementations of algorithms for this task exist.
A ground-breaking result of two teams (\cite{lipshitz2014khovanov,kriz2016khovanov}) states that this invariant can be obtained from a cellular spectrum; which implies, indirectly, that $\mathrm{KH_\bullet}$ has an action of the Steenrod algebra $\cA_p$ for any prime $p$.
My team will focus on the design and implementation of algorithms computing this finer invariant effectively.

\medskip\noindent\textbf{Earlier work}.
Adapting the formulas of \cite{medina2023fast_sq}, Cantero-Mor\'an constructed structure at the chain level inducing the action of $\cA_2$ on $\KH$ \cite{cantero-moran2020khovanov}.
His effective methods recover those of Lipshitz and Sakar, which deal with the computation of $\Sq^1$ and $\Sq^2$ \cite{lipshitz2014steenrod}.
These two operations were implemented by Seed \cite{seed2012khovanov} and, last year, all squares were implemented into \texttt{SAGE} by Milstein using Cantero-Morán's structure \cite{milstein2022khovanov}.
Put together, these developments have effectively provided a complete treatment of all primary operations on $\KH$ at the even prime.

\medskip\noindent\textbf{Goals}.
Provide a complete treatment of primary operations on $\KH$ at all primes.
Explicitly, we will achieve the following goals:

\smallskip\noindent(PhD2a)
Effectively describing structure inducing the action on $\KH$ of $\cA_p$ for any prime p.\par
\smallskip\noindent(PhD2b)
Designing and implementing algorithms for the computation of all Steenrod operations on $\KH$.\par
\smallskip\noindent(PhD2c)
Computing these invariants for $p=3,5,7$ and knots with up to 14 crossing.

\smallskip\noindent This software's primary impact will be in the field of low-dimensional topology, enabling a more precise exploration of intricate properties related to knots and links.

\smallskip\noindent\textbf{Methodology}.
We have already agreed to collaborate with Cantero-Morán, who will serve as a secondary advisor.
Furthermore, the support of ORCCA will ensure the quality of the resulting software.
	% !TEX root = ../proposal.tex

\smallskip
{\centering (3) \textsc{Steenrod barcodes \& persistent cup-length of real-world data} (MSc4-5, PhD4)\par}

\medskip\noindent\textbf{Deliverables}.
a) High-performance implementation of tools for the computation of Steenrod barcodes and persistent cup-length.
b) A catalogue of these for densely sampled molecular conformation spaces.
c) At least six publications in high-impact journals.

\smallskip\textbf{Strategic partnerships}.
As before, the expertise of ORCCA will play a critical role in this endeavor. We will also collaborate with U. Lupo from EPFL, the maintainer of \texttt{giottoTDA}, to seamlessly integrate our newly developed tools into this widely-used topological data analysis (TDA) platform.
Additionally, we will join forces with F.~Mémoli from Ohio State University for the development of \texttt{cuplengther}.

\smallskip\noindent\textbf{Background}.
Persistent homology is a central method in TDA that quantifies the topological features of a dataset at various spatial resolutions, often represented as a so-called \textit{barcode}.
High-performance algorithmic implementations for the computation of barcodes like \texttt{giottoTDA} \cite{medina2021giotto}, \texttt{ripser} \cite{bauer2021ripser}, and \texttt{gudhi} \cite{maria2014gudhi} have significantly contributed to its broad adoption across various scientific disciplines; for a recent survey, see \cite{carlsson2021topological}.
However, it is crucial to acknowledge that the basic barcode obtained this way has notable limitations paralleling the inherent issues of (non-persistent) cohomology discussed in (1).
It is natural then to envision enhancing persistent cohomology, without loosing effective computability, with additional structure present in cohomology.
For example, with the action of the Steenrod algebra $\cA_p$ for some $p$ and its ring structure.

\medskip\noindent\textbf{Key idea}.
We will generalize our effective algebro-homotopical constructions from triangulated spaces to nested families of triangulated spaces, the type of objects appearing in the multi-scale analysis of data.

\medskip\noindent\textbf{Methodology}.
We successfully used this idea in \cite{medina2022per_st}.
Starting with my formulas for cup-$i$ products \cite{medina2023fast_sq} we incorporated the action of $\cA_2$ into the persistent pipeline.
The resulting invariant is a sequence of barcodes for $k \geq 0$, where $k = 0$ corresponds to the basic mod 2 barcode.
For $k > 0$, these barcodes are stable and encode generalizations of the self-intersection of cycles on a closed manifold. Importantly, they are computable.
In fact, in collaboration with members of the \texttt{giottoTDA} team, we developed \texttt{steenroder}, a tool for the computation of mod 2 Steenrod barcodes.
This tool played a crucial role in detecting the presence of Steenrod barcodes in real-world data, as detailed in \cite{medina2022per_st}, specifically within the conformation space of the cyclo-octane molecule.
However, the performance of this tool is not yet optimal and it is restricted to Steenrod operations at the even prime.
(\underline{MSc4})
will focus on enhancing the performance of \texttt{steenroder}, which is currently implemented in \texttt{Python}.
Initially, (MSc4) will tackle the problem of identifying cocycle representatives for persistent cohomology that have minimal support, drawing on similar optimization studies for persistent homology representatives \cite{minimal, obayashi2018optimal}.
Subsequently, (MSc4) will implement parallel processing for constructing Steenrod cocycles from these optimized cocycle representatives.
This advancement will significantly expedite the computation of mod 2 Steenrod barcodes, thereby making it feasible to create a comprehensive catalogue of this invariant for molecular conformation spaces.
(\underline{PhD4}) will extend the capabilities of \texttt{steenroder} to support all prime numbers.
Firstly, (PhD4) will utilize the author's formulas for cup-$(p,i)$ products \cite{medina2021may_st}, along with the algorithms of (1), to develop a computational framework for mod $p$ Steenrod barcodes.
Upon completing this initial theoretical groundwork, (PhD4) will proceed to create a high-performance implementation of this new algorithm.
This enhancement will then be integrated into the existing \texttt{steenroder} codebase, thereby extending its utility across primes.
Finally, this tool will be used to enhance the catalogue of topological invariants of molecular conformation spaces.
(\underline{MSc5})
will implement another pioneering enhancement in persistent cohomology: the notion of persistent cup-length, introduced by M\'emoli's team at Ohio State University \cite{memoli2022cup_length}.
This innovative approach incorporates elements of the ring structure in cohomology into the persistent pipeline.
Despite its theoretical promise, no computational implementation currently exists.
Drawing upon our expertise developing high-performance topological software and in collaboration with M\'emoli, (MSc5) will develop \texttt{cuplenghter}, a specialized tool for computing this nuanced invariant.
The tool will subsequently be used to augment the existing catalogue of topological invariants for molecular conformation spaces.

\smallskip\textbf{Significance}.
This project represents a crucial step forward in advancing both the theory and practice of TDA, making persistent cohomology more descriptive and applicable to a broader range of scientific challenges.

\smallskip\textbf{Future work}.
Using the key idea stated above and the accomplishments of both Objective (1) and (3), we will incorporate secondary operations into the persistent pipeline.
	% !TEX root = ../proposal.tex

\smallskip
{\centering (4) \textsc{A bridge between state sum and functorial definitions of TQFTs} (MSc6, PostDoc)\par}

\smallskip Let us return to the trade-off between concreteness and functoriality discussed in the introduction, now in the context of quantum physics.


%\medskip\noindent\textbf{Deliverable}.
%A bridge between the state-sum theoretic and functorial classifications of invertible phases.

\smallskip\noindent\textbf{Background}.
The significance of topology in condensed matter physics is exemplified by the 2016 Nobel Prize in Physics, awarded for ``theoretical discoveries of topological phase transitions and topological phases of matter."
There are primarily two approaches to defining these TQFT's.

\quad On one hand, they are presented using \textit{lattice models}, which can extended to more general space-times using triangulations.
An important class of these are the so-called symmetry-protected invertible topological phases.
Using cochain level structures including cup-$i$ products, and Cartan and Adem coboundaries, these have been classified in low dimensions by Kitaev, Gaiotto--Kapustin, and other physicists \cite{kitaev2009periodic, kapustin2015cobordism, barkeshli2021classification}.
The available examples support their thesis that such classification is controlled by generalized cohomology theories.

\quad On the other hand, the functorial approach to TQFTs, pioneered by Atiyah and Segal \cite{atiyah1988tqft,segal1988conformal}, organizes the discussion around higher categorical versions of functors from the category of manifolds and bordisms to an enhancement of the category of complex vector spaces \cite{baez1995higher,lurie2008classification}.
In the celebrated work of Hopkins and Freed \cite{freed2021reflection}, this functorial viewpoint was used to relate the classification of invertible topological phases to stable homotopy theory \cite{freed2021reflection}, in line with what physicist expected, but the methods are wildly different, and the way the higher categorical theory emerges from the effective methods on discrete space-times is a key question in the field.
We cite from a lecture given by Hopkins in Regensburg [Move to biblio]:
%\footnote{Time 18:00 in \url{https://mediathek2.uni-regensburg.de/playthis/58fddd43542660.38096595}}
``\textit{This story is far from being understood, and there is a big arc of research that goes into trying to really derive this emerging structure from lattice models, and this is very interesting and very difficult area right now."}

\medskip\noindent\textbf{Deliverables}.
At least four publications on high-impact journals.

\medskip\noindent\textbf{Strategic partnership}.
This project will be accomplished in collaboration with researchers from the Perimeter Institute, including L. M\"uller and A. Turzillo.

\medskip\noindent\textbf{Key idea}.
What sets our approach apart from others—and serves as a unifying thread—is our use of Street's free $n$-category on the $n$-simplex \cite{street1987orientals}, commonly referred to as the $n^\th$ oriental $\cO_n$.
The aptness of this object for the stated goal arises from their grounding on both the discrete and higher categorical settings.
Using these canonical categories, one can define the $n$-category generated by a triangulated space $X$ as $\cO(X) = \colim_{\gsimplex\, \downarrow X} \cO_n$.
The connection between Steenrod's cup-$i$ products, used in the lattice formulas, and Street's orientals was established in \cite{medina2020globular}, where the author demonstrated that the intricate structure of $\mathcal{O}_n$ can be naturally derived from the cup-$i$ products of Steenrod.

\medskip\noindent\textbf{Methodology}.
When $X$ is a triangulation of an $n$-manifold $M$, we will enhance the above colimit to reflect tangential structure on $M$, with orientations and framings being the key examples.
We will introduce the notion of \textit{functorial state sum data} as a cone under the the tangentially enhanced Street diagram, whose apex is a sufficiently dualizable $n$-category.
We will then obtain invariants like the partition function using the colimit description of $\cO(X)$.
Preliminary investigations have shown that this method recovers known state sum descriptions of low dimensional TQFTs, including Turaev--Viro's \cite{turaev1992invariants}, and, given the systematic nature of our approach, it will enable us to define new theories, including a non-invertible version of Dijkgraaf--Witten theory \cite{dijkgraaf1990topological}.

\quad The techniques developed in this project will also have an impact on theoretical mathematics.
For example, my student Aaron Huntly (\underline{MSc6}) will prove the following conjecture emanating from our preliminary work on this project: An $A_\infty$-algebra, i.e., a representation of the Stasheff operad \cite{loday2004stasheff} in the category of chain complexes $\Ch$, is the exact same data as a cone over the traditional (i.e. non-tangentially enhanced) Street diagram with apex the desuspension $\rB\Ch$.
%The key insight for this statement comes from convex polytopes, particularly, from a surprising connection between the $n$-simplex and the $(n-2)$-associahedra.

\medskip\noindent\textbf{Significance}.
Having a mature theory of functorial state sum data and a deeper understanding of its relationship to cochain level structures will allow us to bridge the functorial and state sum approaches to the classification of invertible topological phases, and, by doing so, explaining the connection between the approaches taken by, in one hand, Gaiotto--Kapustin and, on the other, Freed--Hopkins.
Additionally, this viewpoint will allow us to define new TQFTs.

\medskip\noindent\textbf{Future work}.
The cochain level structure used by Gaiotto--Kapustin relates to primary and secondary cohomology operations at the even prime only.
We will use the accomplishments of both Objective (1) and (4) to enrich with odd prime cohomology operations the classification of TQFTs.


%(\underline{PostDoc})

%\medskip\noindent\textsc{Functorial state sum data}
%In collaboration with researchers at the Perimeter Institute and a postdoctoral scholar affiliated with both institutions, we will accomplish the following tasks:
%
%\smallskip
%\noindent(3a) Relate lattice theoretic descriptions of TQFTs to Street's orientals.\par
%\noindent(3b) Relate Street's orientals to functorial descriptions of TQFTs.\par
%\noindent(3c) Define new TQFT's.\par
%\noindent(3d) Validate the Freed--Hopkins functorial classification of invertible topological phases.
%
%\medskip\noindent\textbf{Earlier work}.
%The work by Gaiotto, Kapustin, and other physicists in the classification of low-dimensional symmetry protected invertible fermionic topological phases \cite{gaiotto2016spin, barkeshli2021classification} heavily relies on the cup-$i$ products of Steenrod and our formulas for Cartan and Adem coboundaries \cite{medina2020cartan, medina2021adem}.
%In fact, these application in physics served as the primary motivation for our work.
%We also mentioned work by the PI's coauthors Brumfiel and Morgan \cite{brumfiel2016pontrjagin, brumfiel2018pontrjagin}, where they used these ideas to constructed cochain approximations to the Pontryagin dual of the spin bordism spectrum, the type of object that appears in the Freed--Hopkins classification.
%
%The connection between Steenrod's cup-$i$ products and Street's orientals was established in \cite{medina2020globular}, where the PI surprisingly demonstrated that the intricate structure of $\mathcal{O}_n$ can be naturally derived from the cup-$i$ products of Steenrod.
%Using these canonical categories, one can define the $n$-category generated by a simplicial set $X$
%\begin{equation}\label{eq:colimit}
%	\cO(X) = \colim_{\gsimplex\, \downarrow X} \cO_n.
%\end{equation}
%
%\medskip\noindent\textbf{Methodology}.
%When $X$ is a triangulation of an $n$-manifold $M$, we will enhance the above colimit to reflect tangential structure on $M$, with orientations and framings being the key examples.
%We will then introduce the notion of functorial state sum data as a cone under the the tangentially enhanced Street diagram, whose apex is a sufficiently dualizable $n$-category.
%We will then obtain invariants like the partition function using the colimit description of $\cO(X)$.
%Preliminary investigations have shown that this method recovers known state sum descriptions of low dimensional TQFTs, incuding Turaev--Viro's \cite{turaev1992invariants}, and, given the systematic nature of our approach, it will enable us to define new theories, including a non-invertible version of Dijkgraaf-Witten theory \cite{dijkgraaf1990topological}.
%
%(MSc5) The techniques developed in this project will also have an impact on theoretical mathematics.
%For example, the PI's current master student Aaron Huntly will prove the following conjecture emanating from our preliminary work on this project: An $A_\infty$-algebra, i.e., a representation of the Stasheff operad \cite{loday2004stasheff} in the category of chain complexes $\Ch$, is the exact same data as a cone over the traditional (i.e. non-tangentially enhanced) Street diagram with apex the desuspension $\rB\Ch$.
%The key insight for this statement comes from convex polytopes, particularly, from a surprising connection between the $n$-simplex and the $(n-2)$-associahedra.
%
%(PostDoc) Having a mature theory of functorial state sum data and a deeper understanding of its relationship to cochain level structures will allow us to bridge the functorial and state sum approaches to the classification of invertible topological phases, and, by doing so, explaining the connection between the work of Gaiotto--Kapustin and Freed--Hopkins.


%This project will be accomplished in collaboration with researchers at the Perimeter Institute, including Lukas M\"uller and Alex Turzillo.
%With the support of this institution we will also involve a postdoctoral scholar in this project with a background in both topology and physics.

%\medskip\noindent\textsc{Street orientals and functorial field theories} The free $n$-category generated by the standard $n$-simplex is a fundamental object in category theory introduced by Street in \cite{street1987orientals}.
%Together with Lukas M\"uller and Alex Turzillo of the Perimeter Institute we are exploring the use of this concept as the underpin for a connection between functorial description and triangulations.


%For example, fermionic phases protected by a $G$-symmetry are believed to be classified by applying to $BG$ the Pontryagin dual of spin bordism.
%Building on these insights and using a formula introduced in \cite{medina2020cartan}, A.~Kapustin proposed a structural ansatz in low dimensions that G. Brumfiel and J. Morgan verified by constructing cochain models of certain connective covers of said spectrum.
%
%
%In the research conducted by A. Kapustin and colleagues mentioned earlier, Steenrod cup-i products played a vital role in their state sum formulations.

%\subsection{Non-invertible Dijkgraaf--Witten theory}
%
%In traditional Dijkgraaf--Witten theory the fields are principal bundles for a finite
%group $G$, which form a groupoid.
%An alternative, but homotopy equivalent, description of the collection of all fields on a $d$-manifold $M$ is the mapping space $\mathrm{Map}(M, \rB G)$, which is an $\infty$-groupoid.
%Joint work with L. M\"uller and L. Stehouwer moves away from the manifest invertibility of these groupoids by considering a triangulation on $M$ and the free $n$-category structure it generates -- please compare with \cref{ss:nerve} and \cref{ss:polytopes}.
%The resulting more general theories have fields given by $d$-functors from $M$ to appropriate $d$-categorical generalizations of $\rB G$.

%\subsection{Symmetry protected topological phases and cochain constructions} \label{ss:spt phases}
%
%A central problem in physics is to define and understand the moduli ``space'' of quantum systems with a fixed set of invariants, for example their dimension and symmetry type.
%In condensed matter physics, quantum systems are presented using \textit{lattice models} which, intuitively, are given by a Hamiltonian presented as a sum of local terms on a Hilbert space associated to a lattice in $\R^n$.
%We think of these as defined on flat space.
%One such system is said to be \textit{gapped} if the spectrum of the Hamiltonian is bounded away from $0$, and two Hamiltonians represent the same \textit{phase} if there exists a deformation between them consisting only of systems that remain bounded from below.
%An important class of phases are the \textit{invertible} ones, which are those that can be combined with another to give the trivial phase.
%
%Given a lattice model, by means of cellular decompositions and state sum type constructions, one can often compute the associated \textit{partition functions} on spacetime manifolds.
%The fields and actions been expressed using cochain level structure, for example, Stiefel--Whitney cochains, cup-$i$ products and Adem coboundaries.
%Subdivision invariance gives rise to a functorial TQFT, which in the \textit{invertible} case is expected to be controlled by a generalized cohomology theory, as pioneered by Kitaev, Kapustin and other physicists \cite{kitaev2011topological,kapustin2015cobordism}.
%
%The cochain level structure used in the definition of the cellular gauge theory is interpreted from this point of view as describing a cochain model of the Postnikov tower of the relevant spectrum.
%For example, fermionic phases protected by a $G$-symmetry are believed to be classified by applying to $BG$ the Pontryagin dual of spin bordism.
%Building on these insights and using a formula introduced in \cite{medina2020cartan}, A.~Kapustin proposed a structural ansatz in low dimensions that G. Brumfiel and J. Morgan verified by constructing cochain models of certain connective covers of said spectrum.
%
%In the future, the research program presented here will continue deepening the understanding of the discrete and algebraic structures underpinning SPT phases, with the ultimate goal of elucidating the physics content of the complementary viewpoints provided in stable homotopy theory by functorial and effective constructions.

%of developing, alongside the functorial approach to stable homotopy theory, the effective one as well.

%between discrete models, stable homotopy theory, topological field theories.
%This is a long term project, but a first concrete objective is the description of the signature of 4 manifolds through a combinatorial local state sum formula.
%The importance of this question has been highlighted by Peter Teichner.
%Partial results for general $4k$-manifolds have been achieved by Dennis Sullivan and Andrew Ranicki \cite{sullivan1976signature}.

	\newpage
	\sloppy
	\printbibliography
\end{document}

Proposal

Five pages max.

Addressing the points below, describe the proposed research to be supported. Images and graphics are included in the page limit.

Recent progress

Describe your recent progress in research activities related to the proposal; for returning grantees, describe as well the progress attributable to your previous Discovery Grant.

Objectives

Define the short- and long-term objectives of your research program. Note that a research program should have a long-term vision that expands beyond the five years of the Discovery Grant. A single, short-term project or collection of projects does not constitute a research program.

Literature review

Discuss the literature pertinent to the proposal, placing the proposed research in the context of the state of the art.

Methodology

Describe the methods and proposed approach, providing sufficient details to allow the reviewers to assess the feasibility of the research activities.

Considering equity, diversity and inclusion (EDI) in the research process promotes research excellence by making research outcomes more ethically sound, rigorous, reproducible, and useful. It is important to consider EDI through each stage of the research process including, but not limited to, the research questions, design, methodology, analysis, interpretation and dissemination of results, and integrate these considerations where relevant. Consult Equity, diversity and inclusion considerations at each stage of the research process for more information.

Impact

Explain the anticipated significance of the work.

Note:

If the information provided is insufficient, NSERC reserves the right to take appropriate action, such as not soliciting reports from external reviewers or withdrawing the application from the competition.
If relevant to your research, consult NSERC’s Guidelines for the preparation and review of applications in interdisciplinary research and/or Guidelines for the preparation and review of applications in engineering and the applied sciences.
If NSERC determines that the subject matter is outside of its mandate at any time during the review cycle, the application will be rejected. See This link will take you to another Web site Selecting the appropriate federal granting agency for more information.
