% !TEX root = ../proposal.tex

\section*{Introduction} \label{s:introduction}

There is a tense trade-off in algebraic topology having roots reaching back to the beginning of its modern form.
This tension can be illustrated with the concept of cohomology.
The first approaches, dating back to Poincar\'e, are based on the subdivision of a space into simple pieces.
These elementary shapes are made to generate a free graded module and their spatial relations define the differential used to compute cohomology.
This definition makes certain geometric properties of cohomology, for example excision, fairly clear.
Yet, it is not easy to show that a continuous map of spaces induces a map between their associated cohomologies.
The functoriality just alluded to is trivial when defining cohomology in terms of homotopy classes of maps to Eilenberg--MacLane spaces, but the passage to the homotopy category erases geometric and combinatorial information and the resulting definition is not well suited for concretely presented spaces.
The tension this example illustrates manifests itself in many other important contexts, and the trade-off between concreteness and functoriality is as central today as it was almost a century ago.

\section*{Objectives}

My long-term vision is to ease this tension by developing concrete constructions of homotopical invariants currently defined only indirectly; and, by doing so, extending the reach of algebraic topology into fields beyond pure mathematics, including Data Science and Quantum Field Theory.
We will organize the presentation of this program around four interrelated deliverables.

\smallskip
\quad (1) Develop software to compute secondary operations on triangulated spaces.\par
\quad (2) Compute all Steenrod operations on Khovanov homology for knots with up to 14 crossings.\par
\quad (3) Compute the Steenrod barcodes of densely sampled molecular conformation spaces.\par
\quad (4) Relate the sate sum and functorial classifications of invertible topological phases of matter.\par