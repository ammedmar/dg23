% !TEX root = ../proposal.tex

\textsc{\large Effective Algebro-Homotopical Constructions with Applications to Data Science and Quantum Field Theory}
--- \hfill \textsc{\large Anibal M. Medina-Mardones}

\bigskip\textsc{Introduction}.
There is a tense trade-off in algebraic topology having roots reaching back to the beginning of its modern form.
This tension can be illustrated with the concept of cohomology.
The first approaches, dating back to Poincar\'e, are based on the subdivision of a space into simple pieces.
These elementary shapes are made to generate a free graded module and their spatial relations define the differential used to compute cohomology.
This definition makes certain geometric properties of cohomology, for example excision, fairly clear.
Yet, it is not easy to show that a continuous map of spaces induces a map between their associated cohomologies.
The functoriality just alluded to is trivial when defining cohomology in terms of homotopy classes of maps to Eilenberg--MacLane spaces, but the passage to the homotopy category erases geometric and combinatorial information and the resulting definition is not well suited for concretely presented spaces.
The tension this example illustrates manifests itself in many other important contexts, and the trade-off between concreteness and functoriality is as central today as it was almost a century ago.

\smallskip\textsc{Objectives}.
\textit{My long-term vision is to ease this tension by developing concrete constructions of homotopical invariants currently defined only indirectly; and, by doing so, extend the reach of algebraic topology into fields beyond pure mathematics, including Data Science and Quantum Field Theory.}

\smallskip We will organize the presentation of this program around four interrelated objectives.

\smallskip
(1) Compute primary \& secondary cohomology operations of triangulated spaces. \hfill (MSc1-2, PhD1-2)\par
(2) Compute Steenrod operations on Khovanov homology of knots and links. \hfill (MSc3, PhD3)\par
(3) Compute Steenrod barcodes and persistent cup-lengths of real-world datasets. \hfill (MSc4-5, PhD4)\par
(4) Build a bridge between state sum and functorial topological quantum field theories. \hfill (MSc6, PostDoc)\par

%\smallskip
%Each of these tangible contributions will follow from novel understanding of the algebro-homotopical properties of topological spaces, with a strong emphasis on computation and effectiveness.