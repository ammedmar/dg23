% !TEX root = ../proposal.tex

\smallskip
{\centering \underline{\textsc{Compute Steenrod barcodes \& persistent cup-length of real-world data}}\par}
{\centering Objective (3) \hspace*{2cm} --- \hspace*{2cm} (MSc4-5, PhD4)\par}

\smallskip\textbf{Strategic partnerships}.
As before, ORCCA's expertise in high-performance computing will play an important role accomplishing this research objective.
We will also collaborate with U. Lupo from EPFL, the maintainer of \texttt{giottoTDA}, to seamlessly integrate our newly developed tools into this widely-used topological data analysis (TDA) platform.
For the development of \texttt{cuplengther}, we will also join forces with F.~Mémoli from Ohio State University, whose team introduced the persistent cup-length invariant conceptually.

\smallskip\noindent\textbf{Background}.
Persistent homology is a central technique in TDA that quantifies, via the so-called \textit{barcode} invariant, the topological features of a dataset at various spatial resolutions.
High-performance computational tools, like \texttt{giottoTDA} \cite{medina2021giotto}, \texttt{ripser} \cite{bauer2021ripser}, and \texttt{gudhi} \cite{maria2014gudhi}, have significantly contributed to this technique's broad adoption across various scientific disciplines; for a recent survey, see \cite{carlsson2021topological}.
However, it is crucial to acknowledge that the basic barcode has notable limitations paralleling the inherent issues of (non-persistent) cohomology discussed in (1).
It is natural, then, to envision enhancing persistent cohomology, without losing effective computability, with additional structures present in $\coho$.
For example, Steenrod operations or the ring structure.

\medskip\noindent\textbf{Key idea}.
The type of objects appearing in the multi-scale analysis of data are nested families of triangulated spaces.
The key idea used to accomplish this research objective is that our effective algebro-homotopical constructions for triangulated spaces can be adapted to such families.

\medskip\noindent\textbf{Methodology}.
We successfully used this idea in \cite{medina2022per_st}.
Starting with my formulas for cup-$i$ products \cite{medina2023fast_sq} we incorporated the action of Steenrod squares into the persistent pipeline.
The resulting invariant is a sequence of barcodes for $k \geq 0$, where $k = 0$ corresponds to the basic mod 2 barcode.
These barcodes are stable and encode generalizations of the self-intersection of cycles on a closed manifold.
Moreover, these barcodes are computable.
In fact, in collaboration with members of the \texttt{giottoTDA} team, we developed \texttt{steenroder}, a tool for the computation of mod 2 Steenrod barcodes.
This tool played a crucial role in detecting the presence of Steenrod barcodes in real-world data, as detailed in \cite{medina2022per_st}, specifically within a dense sampling of the conformation space of the cyclo-octane molecule.
However, the performance of this tool is not yet optimal and it is restricted to Steenrod operations at the even prime.
(\underline{MSc4}), my current student Jianyin Lyu, will focus on enhancing the performance of \texttt{steenroder}, which is currently implemented in \texttt{Python}.
Initially, (\underline{MSc4}) will tackle the problem of identifying cocycle representatives for persistent cohomology that have minimal support, drawing on similar optimization studies for persistent homology representatives \cite{obayashi2018optimal}.
Subsequently, (\underline{MSc4}) will implement parallel processing for constructing Steenrod cocycles from these optimized cocycle representatives.
This advancement will significantly accelerate the computation of mod~2 Steenrod barcodes, thereby making it feasible to create a comprehensive catalogue of this invariant for molecular conformation spaces.

\quad The task of (\underline{PhD4}) will be to extend the capabilities of \texttt{steenroder} to support all prime numbers.
Firstly, (\underline{PhD4}) will utilize the author's formulas for cup-$(p,i)$ products \cite{medina2021may_st}, along with the algorithms of (1), to develop a computational framework for mod $p$ Steenrod barcodes.
Upon completing this initial theoretical groundwork, (\underline{PhD4}) will proceed to create a high-performance implementation of this new algorithm.
This enhancement will then be integrated into the existing \texttt{steenroder} codebase, thereby extending its utility across primes.
Finally, this tool will be used to enhance the catalogue of topological invariants of molecular conformation spaces.

\quad In parallel, (\underline{MSc5}) will implement another pioneering enhancement in persistent cohomology: the notion of persistent cup-length, introduced by M\'emoli's team at Ohio State University \cite{memoli2022cup_length}.
This innovative approach incorporates elements of the ring structure in cohomology into the persistent pipeline.
Despite its theoretical promise, no computational implementation currently exists.
Drawing upon our expertise in the development of high-performance topological software, and in collaboration with M\'emoli, (\underline{MSc5}) will develop \texttt{cuplenghter}, a specialized tool for computing this nuanced invariant.
The tool will subsequently be used to augment the existing catalogue of topological invariants for molecular conformation spaces.

\smallskip\textbf{Significance}.
This research objective represents a crucial step forward in advancing both the theory and practice of TDA, making persistent cohomology more descriptive without compromising its broad scientific applicability.
Making \texttt{steenroder} and \texttt{cuplengther} available to the community will amplify the impact of these ideas, expanding the use of algebraic topology in practical applications.
Using the key idea stated above and the accomplishments of Objective (1), in future years we will incorporate secondary cohomology operations into the persistent pipeline.