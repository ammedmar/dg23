% !TEX root = ../proposal.tex

\smallskip
{\centering (3) \textsc{Steenrod barcodes \& persistent cup-length of real-world data} (MSc4-5, PhD4)\par}

\medskip\noindent\textbf{Deliverables}.
a) High-performance implementation of tools for the computation of Steenrod barcodes and persistent cup-length.
b) A catalogue of these for densely sampled molecular conformation spaces.
c) At least six publications in high-impact journals.

\smallskip\textbf{Strategic partnerships}.
As before, the expertise of ORCCA will play a critical role in this endeavor. We will also collaborate with U. Lupo from EPFL, the maintainer of \texttt{giottoTDA}, to seamlessly integrate our newly developed tools into this widely-used topological data analysis (TDA) platform.
Additionally, we will join forces with F.~Mémoli from Ohio State University for the development of \texttt{cuplengther}.

\smallskip\noindent\textbf{Background}.
Persistent homology is a central method in TDA that quantifies the topological features of a dataset at various spatial resolutions, often represented as a so-called \textit{barcode}.
High-performance algorithmic implementations for the computation of barcodes like \texttt{giottoTDA} \cite{medina2021giotto}, \texttt{ripser} \cite{bauer2021ripser}, and \texttt{gudhi} \cite{maria2014gudhi} have significantly contributed to its broad adoption across various scientific disciplines; for a recent survey, see \cite{carlsson2021topological}.
However, it is crucial to acknowledge that the basic barcode obtained this way has notable limitations paralleling the inherent issues of (non-persistent) cohomology discussed in (1).
It is natural then to envision enhancing persistent cohomology, without loosing effective computability, with additional structure present in cohomology.
For example, with the action of the Steenrod algebra $\cA_p$ for some $p$ and its ring structure.

\medskip\noindent\textbf{Key idea}.
We will generalize our effective algebro-homotopical constructions from triangulated spaces to nested families of triangulated spaces, the type of objects appearing in the multi-scale analysis of data.

\medskip\noindent\textbf{Methodology}.
We successfully used this idea in \cite{medina2022per_st}.
Starting with my formulas for cup-$i$ products \cite{medina2023fast_sq} we incorporated the action of $\cA_2$ into the persistent pipeline.
The resulting invariant is a sequence of barcodes for $k \geq 0$, where $k = 0$ corresponds to the basic mod 2 barcode.
For $k > 0$, these barcodes are stable and encode generalizations of the self-intersection of cycles on a closed manifold. Importantly, they are computable.
In fact, in collaboration with members of the \texttt{giottoTDA} team, we developed \texttt{steenroder}, a tool for the computation of mod 2 Steenrod barcodes.
This tool played a crucial role in detecting the presence of Steenrod barcodes in real-world data, as detailed in \cite{medina2022per_st}, specifically within the conformation space of the cyclo-octane molecule.
However, the performance of this tool is not yet optimal and it is restricted to Steenrod operations at the even prime.
(\underline{MSc4})
will focus on enhancing the performance of \texttt{steenroder}, which is currently implemented in \texttt{Python}.
Initially, (MSc4) will tackle the problem of identifying cocycle representatives for persistent cohomology that have minimal support, drawing on similar optimization studies for persistent homology representatives \cite{minimal, obayashi2018optimal}.
Subsequently, (MSc4) will implement parallel processing for constructing Steenrod cocycles from these optimized cocycle representatives.
This advancement will significantly expedite the computation of mod 2 Steenrod barcodes, thereby making it feasible to create a comprehensive catalogue of this invariant for molecular conformation spaces.
(\underline{PhD4}) will extend the capabilities of \texttt{steenroder} to support all prime numbers.
Firstly, (PhD4) will utilize the author's formulas for cup-$(p,i)$ products \cite{medina2021may_st}, along with the algorithms of (1), to develop a computational framework for mod $p$ Steenrod barcodes.
Upon completing this initial theoretical groundwork, (PhD4) will proceed to create a high-performance implementation of this new algorithm.
This enhancement will then be integrated into the existing \texttt{steenroder} codebase, thereby extending its utility across primes.
Finally, this tool will be used to enhance the catalogue of topological invariants of molecular conformation spaces.
(\underline{MSc5})
will implement another pioneering enhancement in persistent cohomology: the notion of persistent cup-length, introduced by M\'emoli's team at Ohio State University \cite{memoli2022cup_length}.
This innovative approach incorporates elements of the ring structure in cohomology into the persistent pipeline.
Despite its theoretical promise, no computational implementation currently exists.
Drawing upon our expertise developing high-performance topological software and in collaboration with M\'emoli, (MSc5) will develop \texttt{cuplenghter}, a specialized tool for computing this nuanced invariant.
The tool will subsequently be used to augment the existing catalogue of topological invariants for molecular conformation spaces.

\smallskip\textbf{Significance}.
This project represents a crucial step forward in advancing both the theory and practice of TDA, making persistent cohomology more descriptive and applicable to a broader range of scientific challenges.

\smallskip\textbf{Future work}.
Using the key idea stated above and the accomplishments of both Objective (1) and (3), we will incorporate secondary operations into the persistent pipeline.