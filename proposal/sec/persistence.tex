% !TEX root = ../proposal.tex

\section*{(2) Refinements and Novel Applications of Persistent Cohomology \\ (PhD3, MSc3, MSc4)}

Persistent (co)homology is a powerful technique used in the field of topological data analysis.
It serves as a valuable tool for analyzing and extracting topological features and structures from data, especially in the context of point clouds or other high-dimensional data representations.
The core steps i the persistent (co)homology pipeline are the following:
i) Through various standard techniques, construct from the given data triangulated spaces with a filtration
\[
X_0 \subseteq X_1 \subseteq X_2 \subseteq \dots \subseteq X_N \,.
\]
ii) Apply the (co)homology functor to obtain a diagram
\[
\rH^\bullet X_0 \leftarrow \rH^\bullet X_1 \leftarrow \dots \leftarrow \rH^\bullet X_N \,.
\]
iii) Summarize the resulting sequence of homologies into a computationally efficient invariant known as the ``barcode" of the filtration.
A key feature of persistent (co)homology is the stability property, which asserts that, with appropriate metrics, the passage to barcodes maintains bounded distortion.

Numerous high-performance implementations of algorithms for computing barcodes are available, such as \texttt{giotto-tda} \cite{medina2021giotto}, \texttt{ripser} \cite{bauer2021ripser}, and \texttt{gudhi} \cite{gudhi}.
This is one of the main reasons explaining the significant impact this technique is having across scientific domains.
Consult \cite{carlsson2021topological} for a fairly recent survey.
However, it is important to recognize that the classical barcode, despite its utility, is not without substantial limitations, as it does not encode the way bars interact with each other.
These limitations extend from those associated with (non-persistent) cohomology discussed in the previous section.

The second goal of this research program is to develop finer computable invariants for persistent (co)homology, provide high-performance implementations for the computation of new and existing invariants, and identify domains of applicability where this techniques will proved domain specific insights.
These software tools will enable the analysis of significantly larger datasets, offering ample examples to interpret these innovative topological signature of data within domain-specific contexts, such as molecular chemistry and neuroscience.

\medskip\noindent\textbf{Earlier work}.
Using the novel algorithm for the calculation of Steenrod squares introduced in \cite{medina2023fast_sq}, we effectively integrated finer information encoded by Steenrod operations, specifically, the action of $\cA_2$ on mod 2 cohomology, into the persistence (co)homology pipeline \cite{medina2022per_st}.
The resulting invariant takes the form of a sequence of barcodes for $k \geq 0$, with $k = 0$ recovering the traditional mod 2 barcode. For $k > 0$, these barcodes are also stable and encode generalizations of self-intersection of cycles on a closed manifold. Importantly, they are computable.
In fact, in collaboration with members of the \giottoTDA\ team, we developed \texttt{steenroder}, a tool for the computation of Steenrod barcodes.
This tool played a crucial role in detecting the presence of Steenrod barcodes in real-world data, as detailed in \cite{medina2022per_st}, specifically within the conformation space of the cyclo-octane molecule.
However, it is worth noting that the performance of this tool is not yet optimal.
In the following section, we will describe how to improve its efficiency.

In another significant conceptual advance towards enhancing persistent (co)homology with finer computable invariants, as discussed in \cite{memoli2022cup_length}, the authors introduced the notion of the cup-length invariant. This innovation incorporates a portion of the ring structure on cohomology into the persistent homology pipeline.
More precisely, the cup-length of a graded ring is defined as the largest non-negative integer $\ell$ for which there exist positive-dimensional homogeneous elements $a_1, \dots, a_\ell$ within the ring such that their product, $a_1 \dots a_\ell$, is nonzero. The persistent version of this invariant is defined as a function over the filtration parameter.
Regrettably, it is worth noting that as of now, there is no existing implementation of their algorithm.

\medskip\noindent\textsc{High-performance \texttt{steenroder} and \texttt{cuplengther}}
There are three steps in the computation of Steenrod barcodes:
i) The first step involves computing the traditional barcode. This is accomplished through a matrix reduction routine that has been extensively optimized through the collaborative efforts of multiple researchers.
For a cross-software performance comparison one can consult \cite{otter2017roadmap}.
ii)~The second step utilizes a representative cocycle $\alpha$ for each bar to construct a representative cocycle $\beta$ for $\Sq^k[\alpha]$.
This is done using the algorithm introduced in \cite{medina2023fast_sq}.
iii)~Finally, the third step involves another matrix reduction procedure that examines the linear independency of the resulting Steenrod cocycles.
The second step, the current bottleneck of \texttt{steenroder}, is quadratic on the length of the cocycle representative $\alpha$.
The delays are caused by overly representatives with large support, and the lack of parallelization of the routine across cocycles.

Additionally, advances on cocycle replacement will allow for the efficient implementation of an algorithm computing the persistent cup-length invariant.

\medskip\noindent\textbf{Goals}.
Improve the performance of \texttt{steenroder}, implement \texttt{cuplengther}, and test them on real-world data.
Explicitly, my team will complete the following tasks:

\smallskip\noindent(MSC3a)
Parallelizing the construction of Steenrod cocycles across cocycle representatives.

\smallskip\noindent(PhD3a)
Replacing naively obtained cocycles with alternatives having smaller support.

\smallskip\noindent(PhD3b) Implementing \texttt{cuplengther}, a tool for the computation of persistent cup-length.

\smallskip\noindent(PhD3c)
Obtaining a catalogue of topological signatures of densely sampled molecular conformation spaces.

\smallskip\noindent\textbf{Methodology}.
The software development components of this project will be conducted in collaboration with ORCCA for overall support.
Additionally, for the specific application in chemistry, we will partner with the Theory and Computation research group at Western.
