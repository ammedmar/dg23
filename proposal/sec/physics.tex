% !TEX root = ../proposal.tex

\smallskip
{\centering \underline{\textsc{Build a bridge between state sum and functorial definitions of TQFTs}}\par}
{\centering Objective (4) \hspace*{2cm} --- \hspace*{2cm} (MSc3, PostDoc)\par}

\medskip Let us return to the trade-off between constructiveness and functoriality discussed in the introduction, now in the context of quantum physics.

\smallskip\noindent\textbf{Background}.
The significance of topology in condensed matter physics is exemplified by the 2016 Nobel Prize in Physics, awarded for ``theoretical discoveries of topological phase transitions and topological phases of matter."
There are primarily two approaches to defining these TQFT's.
On one hand, they are presented using \textit{lattice models}, which can extended to more general space-times using manifold triangulations.
An important class of these are the so-called \textit{symmetry-protected invertible topological phases}.
Using cochain level structures including cup-$i$ products, and Adem and Cartan coboundaries, these have been classified in low dimensions by Kitaev, Gaiotto--Kapustin, and other physicists \cite{kitaev2009periodic, kapustin2015cobordism, barkeshli2022classification}.
Their examples support the ansatz that such classification is controlled by generalized cohomology theories \cite{adams1995stable}.

\quad On the other hand, the functorial approach to TQFTs, pioneered by Atiyah and Segal \cite{atiyah1988tqft}, organizes the discussion around a higher categorical version of functors from the category of manifolds and bordisms to the category of complex vector spaces \cite{baez1995higher}.
In the celebrated work of Hopkins and Freed \cite{freed2021reflection}, this functorial viewpoint was used to relate the classification of invertible topological phases to stable homotopy theory \cite{freed2021reflection}, in line with what physicists expected, but the methods are wildly different, and the way the higher categorical theory emerges from the effective methods on discrete space-times is a key question in the field.
We quote from a lecture given by Hopkins in Regensburg \cite{hopkins2023higher}:
``\textit{This story is far from being understood, and there is a big arc of research that goes into trying to really derive this emerging structure from lattice models, and this is very interesting and very difficult area right now."}

\smallskip\noindent\textbf{Strategic partnership}.
This research objective will be accomplished in collaboration with researchers from the Perimeter Institute, including L. M\"uller and A. Turzillo.

\smallskip\noindent\textbf{Key idea}.
What sets our approach apart from others—and serves as a unifying thread—is our use of Street's free $n$-category on the $n$-simplex \cite{street1987orientals}, commonly referred to as the $n^\th$ oriental $\cO_n$.
The aptness of this object for the stated goal arises from its grounding on both the discrete and higher categorical settings.
% grounded in
Using these canonical categories, one can define the $n$-category generated by a triangulated space $X$ as $\cO(X) = \colim_{\gsimplex\, \downarrow X} \cO_n$.
The connection between Steenrod's cup-$i$ products, used in the lattice formulas, and Street's orientals was established in \cite{medina2020globular}, where the author demonstrated that the intricate structure of $\mathcal{O}_n$ can be naturally derived from the cup-$i$ products of Steenrod.

\smallskip\noindent\textbf{Methodology}.
When $X$ is a triangulation of an $n$-manifold $M$, we will enhance the above colimit to reflect tangential structure on $M$, with orientations and framings being the key examples.
We will introduce the notion of \textit{functorial state sum data} as a cone under the tangentially enhanced Street diagram, whose apex is a sufficiently dualizable $n$-category.
We will then obtain invariants like the partition function using the colimit description of $\cO(X)$.
Preliminary investigations have shown that this method recovers known state sum descriptions of low dimensional TQFTs, including Turaev--Viro's, and, given the systematic nature of our approach, it will enable us to define new theories, including a non-invertible version of Dijkgraaf--Witten theory.

\quad The techniques developed to accomplish this research objective will also have an impact on theoretical mathematics.
For example, my student Aaron Huntley (\underline{MSc6}) will prove the following conjecture emanating from our preliminary work towards this objective: An $A_\infty$-algebra, i.e., a representation of the Stasheff operad in the category of chain complexes $\Ch$, is the exact same data as a cone over the traditional (i.e. non-tangentially enhanced) Street diagram with apex the desuspension $\rB\Ch$.

\smallskip\noindent\textbf{Significance}.
Achieving a comprehensive theory of functorial state sum data, along with a deeper understanding of its connection to cochain-level structures, will enable us to reconcile the functorial and state-sum approaches to classifying invertible topological phases.
This will not only allow for the definition of new TQFTs, but also elucidate the divergent paths taken by Gaiotto--Kapustin and Freed--Hopkins in their respective approaches.