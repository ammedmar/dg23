% !TEX root = ../proposal.tex

\section*{Background}

As mentioned in the introduction, Poincaré's definition of the cohomology of a space $X$ relies on its cochains, denoted simply as $\cochains$, which arise from a cellular decomposition of $X$.
This construction is local in nature, allowing for a piecewise analysis of $X$, and is thus well-suited for concrete computations.
The computer algebra system (CAS) \texttt{SageMath}, among others, is a tool where this approach to cohomology is implemented.
Unfortunately, cohomology—or equivalently, the quasi-isomorphism type of $\cochains$—omits crucial homotopical information.
As an illustrative example, consider the spaces $\bC P^2$ and $S^2 \vee S^4$, the union of a 2- and a 4-sphere over a point.
Despite their cohomology groups being isomorphic, these spaces are not homotopy equivalent.
To distinguish them, one can employ the Alexander--Whitney product in $\cochains$ to reveal that their ring structures in cohomology are not isomorphic \cite{alexander1936ring, whitney1938products}.
This product-enhanced version of $\cochains$ retains the original's computability and is also supported in \texttt{SageMath}.
Considering the suspension of these spaces leads to two non-homotopic spaces $\Sigma(\mathbb{C} P^2)$ and $\Sigma(S^2 \vee S^4)$ whose cohomologies are isomorphic as graded rings.
To distinguish them we can consider Steenrod squares on $\rH^\bullet$ which are effectively computable through the cup-$i$ products \cite{steenrod1947products}, a structure on $\cochains$ also implemented in \texttt{SageMath}, which generalizes the Alexander-Whitney structure.
These enhancements on $\cochains$ would not suffice to distinguish a similar comparison involving a quaternionic projective space.
In that case, we can use odd prime power operations, but these are not available on mainstream computer algebra systems yet.

\smallskip
A ground-breaking result in modern homotopy theory, due to Mandell \cite{mandell2006homotopy_type}, is that, under certain finiteness assumptions, all the homotopical information of $X$ is captured by an enhancement of the cup-$i$ product structure on $\cochains$.
This complete algebro-homotopical invariant, known as an $E_\infty$-algebra, is the conceptual cornerstone of our program, since, after work by the author and others, it can be given a concrete local description (\cite{mcclure2003multivariable,berger2004combinatorial,medina2020prop1}).

%%As mentioned in the introduction, Poincar\'e's definition of the cohomology of a space $X$ relies on the cochains $\cochains$ of a cellular decomposition of $X$.
%%This construction is local and well-suited for concrete computations.
%%Unfortunately, cohomology, or equivalently the quasi-isomorphism type of $\cochains$, forgets much homotopical information.
%%For example, the cohomology groups of $\mathbb{C} P^2$ and $S^2 \vee S^4$, the union of a $2$- and a $4$-sphere over a point, are isomorphic despite these spaces not being homotopy equivalent.
%%By choosing triangulations for these spaces, the Alexander--Whitney product can be used to show that the ring structure in cohomology distinguishes them.
%%An algorithm for this computation is available for example on \texttt{SageMath}.
%
%
%A fundamental step in the direction of our goal is the description of a ring structure on $\rH^\bullet$ using the Alexander--Whitney product on simplicial cochains $\cochains$ \cite{alexander1936ring, whitney1938products}.
%Let us illustrate with an example the additional information captured by this ring structure.
%The cohomology groups of $\mathbb{C} P^2$ and $S^2 \vee S^4$, the union of a $2$- and a $4$-sphere over a point, are isomorphic despite these spaces not being homotopy equivalent.
%By choosing triangulations for these spaces, the Alexander--Whitney product can be used to show that the ring structure in cohomology distinguishes them.
%An algorithm for this type of computation is available for example on \texttt{SAGE}.
%Considering the suspension of these spaces leads to two non-homotopic spaces $\Sigma(\mathbb{C} P^2)$ and $\Sigma(S^2 \vee S^4)$ whose cohomologies are isomorphic as graded rings.
%Distinguishing these effectively lead us to a novel step in the direction of our goal.
%Let us consider a prime $p$.
%Steenrod made $\rH^\bullet$ it into a module over his algebra $\cA_p$ \cite{steenrod1962cohomology}.
%This structure is explicitly induced from the $E_\infty$-algebra structure on simplicial cochains by means of the cup-$(p,i)$ products.
%These generalize to odd primes the cup-$i$ products of Steenrod, and were introduced and implemented, using effective versions of May's operadic methods \cite{may1970general}, in \cite{medina2021may_st,medina2021comch}.
%The action of $\cA_2$ suffices to distinguish $\Sigma(\mathbb{C} P^2)$ and $\Sigma(S^2 \vee S^4)$, but and odd prime will be needed for similar comparisons using the suspension of quaternionic projective spaces, for example.
%
%
%A ground-breaking result in modern homotopy theory, due to Mandell \cite{mandell2006homotopy_type}, is that, under certain finiteness assumptions, all the homotopical information of $X$ is captured by a so-called $E_\infty$-algebra structure on $\cochains$.
%This complete algebro-homotopical invariant is the conceptual cornerstone of our program, since, after work by the author and others, it can be given a concrete local description (\cite{mcclure2003multivariable,berger2004combinatorial,medina2020prop1}).
%This allows for the effective description of progressively richer homotopical invariants that can be implemented into computer algebra systems (CAS) such as \texttt{ComCH}, \texttt{SageMath}, and \texttt{Maple}.