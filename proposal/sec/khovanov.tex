% !TEX root = ../proposal.tex

\smallskip
{\centering \underline{\textsc{Compute Steenrod operations on Khovanov homology of knots and links}}\par}
{\centering Objective (2) \hspace*{2cm} --- \hspace*{2cm} (MSc3, PhD3)\par}

\smallskip\textbf{Strategic partnerships}.
To ensure the accomplishment of this research objective, we will again leverage ORCCA's vast expertise in software development and high-performance computing.
Additionally, we will continue collaborating with Cantero-Mor\'an, an expert in Khovanov homology.
The same caveat made in (1) about \texttt{Maple} is in place here.

\smallskip\textbf{Background}.
Khovanov homology $\KH$ is a powerful algebraic invariant of knots and links which refines the Jones polynomial \cite{khovanov2000khovanov}.
It is effectively computable from a chain complex associated with a knot diagram, and several implementations of this routine exist.
A ground-breaking result of two teams (\cite{lipshitz2014khovanov, kriz2016khovanov}) proved that this invariant can be obtained from a cellular spectrum in the sense of stable homotopy theory, which implies, indirectly, that $\mathrm{KH_\bullet}$ has an action of Steenrod operations for any prime $p$.

\smallskip\textbf{Key idea}.
The Khovanov spectrum of a link or knot can be modelled by an object fairly similar to a triangulated space, a semi-simplicial object in the Burnside 2-category \cite{lawson2020khovanov}.
The key idea that we will use to accomplish this research objective is that our effective algebro-homotopical constructions for triangulated spaces can be adapted to such objects.

\smallskip\noindent\textbf{Methodology}.
This idea was pioneered by Cantero-Mor\'an in \cite{cantero-moran2020khovanov} where he used my formulas for cup-$i$ products \cite{medina2023fast_sq} to construct Steenrod squares on $\KH$.
His formulas generalize those of Lipshitz and Sakar \cite{lipshitz2014steenrod}, which dealt with the computation of $\Sq^1$ and $\Sq^2$ only.
These two operations were implemented by Seed \cite{seed2012khovanov} and, last year, all Steenrod squares were implemented in \texttt{SAGE} by Milstein using Cantero-Mor\'an's effective constructions \cite{milstein2022khovanov}.
These developments have provided a complete treatment of Steenrod operations on $\KH$ at the even prime.
(\underline{MSc3}) will conduct a performance profiling of Milstein's Steenrod squares computation tool, which is currently implemented in \texttt{Python}.
Following this analysis, we will employ optimization strategies such as code parallelization, porting the code to a more performance-efficient programming language, and then binding these optimized routines back into the original \texttt{Python} codebase.
This optimization will enable (\underline{MSc3}) to compute Steenrod squares for all prime knots with up to 18 crossings, surpassing Milstein's existing state-of-the-art analysis, which is restricted to 14 crossings.

\quad The task of (\underline{PhD3}) will be to provide a similarly comprehensive analysis of Steenrod operations on $\KH$ at odd primes.
The initial phase involves adapting my cup-$(p,i)$ products \cite{medina2021may_st} to the semi-simplicial Burnside context.
Preliminary investigations conducted with Cantero-Mor\'an indicate that this adaptation is both intricate and intellectually stimulating, revealing connections to Tate cohomology and cochain level stability.
Subsequently, (\underline{PhD3}) will build upon the codebases developed by Milstein and (\underline{MSc3}) to create a high-performance computational tool for evaluating general Steenrod operations on $\KH$.
This tool will enable the computation of these topological invariants for all prime knots with up to 18 crossings.

\smallskip\textbf{Significance}.
Our work will introduce and develop innovative computational tools in the effective study of knots and links and, with them, substantially advance the current state-of-the-art in this field by obtaining a catalogue of Steenrod operations for prime knots with up to 18 crossings.
Using the key idea stated above and the accomplishment of Objective (1), in the following 5 years of my program we will effectively incorporate secondary cohomology operations to the study of knots and links.