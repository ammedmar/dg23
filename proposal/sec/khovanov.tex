% !TEX root = ../proposal.tex

\smallskip
{\centering (2) \textsc{Steenrod operations on Khovanov homology of knots and links} (MSc3, PhD3)\par}

\smallskip\textbf{Deliverables}.
a) Software computing Steenrod operations on Khovanov homology over any prime.
b) A comprehensive catalogue of these topological invariants for knots with up to 18 crossings.
c) At least four publications in high-impact journals.

\smallskip\textbf{Strategic partnerships}.
To ensure the success of this project, we will again leverage ORCCA's vast expertise in software development and high-performance computing.
Additionally, we will continue to collaborate with Cantero-Mor\'an, who is an expert in Khovanov homology.
The same caveat made in (1) about \texttt{Maple} is in place here.

\smallskip\textbf{Background}.
Khovanov homology $\KH$ is a powerful algebraic invariant of knots and links which refines the Jones polynomial \cite{khovanov2000khovanov}.
It is effectively computable from a chain complex associated with a knot diagram, and several implementations of this routine exist.
A ground-breaking result of two teams (\cite{lipshitz2014khovanov,kriz2016khovanov}) proved that this invariant can be obtained from a cellular spectrum in the sense of stable homotopy theory; which implies, indirectly, that $\mathrm{KH_\bullet}$ has an action of the Steenrod algebra $\cA_p$ for any prime $p$.

\smallskip\textbf{Key idea}.
The Khovanov spectrum of a link or knot can be modelled by an object that is fairly similar to a triangulated space, a semi-simplicial object in the Burnside category.
We will generalize our effective algebro-homotopical constructions from tringulated spaces to these type of objects,

\smallskip\noindent\textbf{Methodology}.
This key idea was pioneered by Cantero-Mor\'an in \cite{cantero-moran2020khovanov}.
He used my formulas for cup-$i$ products \cite{medina2023fast_sq} to effectively construct Steenrod squares on Khovanov homology.
His formulas generalize those of Lipshitz and Sakar, which dealt with the computation of $\Sq^1$ and $\Sq^2$ \cite{lipshitz2014steenrod} only.
These two operations were implemented by Seed \cite{seed2012khovanov} and, last year, all squares were implemented into \texttt{SAGE} by Milstein using Cantero-Mor\'an's effective constructions \cite{milstein2022khovanov}.
These developments have provided a complete treatment of Steenrod operations on $\KH$ at the even prime.
(\underline{MSc3}) will conduct a performance profiling of Mildstein's Steenrod squares computation tool, which is currently implemented in \texttt{Python}.
Following this analysis, we will employ optimization strategies such as code parallelization, porting the code to a more performance-efficient programming language, and then binding these optimized routines back into the original \texttt{Python} codebase.
This optimization will enable (MSc3) to compute Steenrod squares for all prime knots with up to 18 crossings, surpassing Mildstein's existing state-of-the-art analysis, which is restricted to 14 crossings.
(\underline{PhD3}) will provide a comprehensive analysis of Steenrod operations on $\KH$ for odd primes.
The initial phase involves adapting my cup-$(p,i)$ products \cite{medina2021may_st}, originally developed for triangulated spaces, to fit the context of semi-simplicial objects in the Burnside category.
Preliminary investigations we have conducted with Cantero-Mor\'an indicate that this adaptation is both intricate and intellectually stimulating, revealing connections to Tate cohomology and cochain level stability.
Subsequently, (PhD3) will build upon the codebases developed by Mildstein and (MSc3) to create a high-performance computational tool for evaluating Steenrod operations on $\KH$.
This tool will enable the computation of these topological invariants for all prime knots with up to 18 crossings.

\smallskip\textbf{Significance}.
Our work will provide a comprehensive framework for computing Steenrod operations on Khovanov homology across both even and odd primes.
This fills an existing gap in the literature and gives researchers a new, effective toolkit for exploring algebraic structures associated with knot invariants.
Overall, the project promises to elevate the state of the art in computational topology, extending its reach and making it more accessible for investigations in both pure and applied mathematical sciences.

\smallskip\textbf{Future work}.
Using the key idea stated above and the accomplishments of both Objective (1) and (2), we will incorporate secondary operations into the analysis of knots and links.