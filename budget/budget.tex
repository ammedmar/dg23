\documentclass[12pt]{article}
\usepackage{amsmath}
\usepackage{microtype}
\usepackage{amssymb}
\usepackage{mathtools}
\usepackage{tikz-cd}
\usepackage{mathbbol} % changes \mathbb{} and adds more support
\usepackage{csquotes}
\usepackage{enumitem}
\usepackage{setspace} % For line spacing
\usepackage{mathptmx} % Times New Roman for both text and math
\usepackage[scaled=.92]{helvet} % Helvetica, scaled to match Times New Roman
\usepackage{layout}

% set up 12pt Times New Roman
\renewcommand{\rmdefault}{ptm}
\renewcommand{\sfdefault}{phv}
\renewcommand{\ttdefault}{pcr}

%\usepackage{layout}
\pagestyle{empty}

\setlength{\parindent}{0pt}
\usepackage[letterpaper,
	margin=1.87cm,
	hoffset=0cm,
	voffset=0cm,
	headheight=0cm,
	headsep=0cm,
	marginparwidth=0cm,
	marginparsep=0cm,
	footskip=0cm]{geometry} % Sets paper size and margins

% line spacing
%\onehalfspacing % Alternatively, you can use \singlespacing to try to meet the "no more than six lines per inch" requirement

% bibliography
\usepackage[
	backend=biber,
	style=numeric, % alphabetic, numeric, remove sorting=none
	sorting=nyt,
	backref=false,
	url=false,
	doi=false,
	isbn=false,
	eprint=false]{biblatex}

\setlength{\bibitemsep}{0.0cm}
\renewbibmacro{in:}{}  % don't display "in:" before the journal name
\AtEveryBibitem{\clearfield{pages}}  % don't show page numbers

% Customize subsection title format
\usepackage{titlesec}
\titleformat{\subsection}[runin]
{\normalfont\bfseries}
{\thesubsection.}
{1em}
{}
[:] % Append a colon at the end

\begin{document}
	\begin{tikzpicture}[remember picture, overlay]
		\node [anchor=north east, inner sep=0.5cm] at (current page.north east)
		{\large Anibal M. Medina-Mardones};
	\end{tikzpicture}
	\vspace*{-20pt}
	\begin{center}
		\huge  Budget Justification
	\end{center}

	{\centering\Large Grand Total \hfill \$354,000}\medskip
	\par This budget has been meticulously crafted to align with the research objectives outlined in our proposal and is critical for our program of research's successful completion.
	We will manage the funds judiciously to maximize both the impact of our research and the preparation of Highly Qualified Personnel in the domains of applied and theoretical topology.
	All values are presented in CAD.

	\bigskip{\centering\Large Personnel Total \hfill \$330,000}

	\smallskip The HQP identified here will focus 100\% of their research time on my Discovery Grant research program.
	Please consult Table 1 for a decription of our personnel across objectives through this five year budget.\medskip

	\medskip{\centering\large MSc Students (5) \hfill \$75,000}\medskip

	\noindent\textit{Justification}:
	The MSc students will be responsible for most of the software development of Objectives (1) and (2) of our proposal. They will also contribute to writing research papers and will be co-authors on at least one paper during the grant period.
	The budget will cover their stipends, in alignment with Western's Mathematics Department guidelines for graduate student compensation (\$7,500 per student per year for a total of \$15,000 for their entire program).
	We mention that, although the program requires 6 MSc students, I request funds for 5 since the first year of two have already been covered using my start up grant. I have ensured there is no overlap in expenses.

	\medskip{\centering\large PhD Students (4) \hfill \$150,000}\medskip

	\noindent\textit{Justification}:
	PhD students will take on more specialized roles in the research program, focusing on deeper challenges in Objectives (1) and (2) of our proposal. Their role will be to provide solutions to hard mathematical problems and guide MSc students through basic topological and algorithmic concepts. They will also be responsible for developing and implementing algorithms. Their roles are crucial for our research program's long-term goals, including publications in high-impact journals. The budget will cover their stipends, in alignment with the Mathematics Department guidelines for graduate student compensation (\$7,500 per student per year for a total of \$37,500 for their entire program).

	\medskip{\centering\large Postdoctoral Researcher (1) \hfill \$45,000}\medskip

	\noindent\textit{Justification}:
	The Postdoctoral Researcher will focus on the deepest of the investigations of our research program, Objective (4), which will bridge the state sum and functorial classification of topological phases.
	To support this, a segment of our budget is allocated to partially cover the Postdoctoral Scholar's salary and benefits (\$15,000 per year for a total of \$45,000 for their entire program) in line with the Mathematics Department guidelines.
	The remaining of portion will be covered through teaching assignments.

	\begin{table}
		\centering
		\begin{tabular}{|c|c|c|c|c|c|c|}
			\hline
			HQP & Objective & Year 1 & Year 2 & Year 3 & Year 4 & Year 5 \\
			\hline
			MSc1& (1) &  & \$7,500 & \$7,500 &  &  \\
			\hline
			MSc2& (1) &  &  &  & \$7,500 & \$7,500 \\
			\hline
			MSc3& (2) &  & \$7,500 & \$7,500 &  &  \\
			\hline
			MSc4& (3) & \$7,500 &  &  &  &  \\
			\hline
			MSc5& (3) &  &  &  & \$7,500 & \$7,500 \\
			\hline
			MSc6& (4) & \$7,500 &  &  &  &  \\
			\hline
			PhD1& (1) & \$7,500 & \$7,500 & \$7,500 & \$7,500 & \$7,500 \\
			\hline
			PhD2& (1) & \$7,500 & \$7,500 & \$7,500 & \$7,500 & \$7,500 \\
			\hline
			PhD3& (2) & \$7,500 & \$7,500 & \$7,500 & \$7,500 & \$7,500 \\
			\hline
			PhD4& (3) & \$7,500 & \$7,500 & \$7,500 & \$7,500 & \$7,500 \\
			\hline
			PostDoc& (4) &  &  & \$15,000 & \$15,000 & \$15,000 \\
			\hline
		\end{tabular}
		\caption{Personnel across objectives through the five year budget}
	\end{table}

	\bigskip{\centering\Large Equipment or Facility \hfill \$24,000}\medskip

	\noindent\textit{Justification}: Considering the critical role of software development and data analysis in our research activities, equipping each HQP with a high-quality laptop is essential for several key reasons. Modern research is computationally intensive and requires robust hardware for efficient data analysis and simulations. Consistency in computational tools among team members also streamlines collaborative efforts and enhances data security. Additionally, having dedicated laptops enables the flexibility for HQP to work remotely, an increasingly important feature in various work configurations. To this end, I have earmarked \$2,000 per team member specifically for laptop acquisition, constituting the full extent of our equipment-related budgetary outlay.

	\newpage
	\bigskip{\centering\Large Materials and supplies \hfill \$10,000}\medskip

	\noindent\textit{Justification}:
	This category allocates \$2000 yearly for software licenses and costs of cloud computing.
	In the modern research landscape, allocating a specific category for software licenses and cloud computing costs in the budget is indispensable for a variety of reasons. Software licenses not only provide access to specialized tools necessary for tasks like statistical analysis, data visualization, and computational modelling but also ensure legal compliance and offer valuable technical support and updates. On the other hand, cloud computing facilitates scalability, allowing our research initiatives to flexibly expand or contract computing resources as needed. Both of these components are crucial for facilitating data analysis, simulations, and collaboration, making them essential items to account for in any well-planned research budget.

	\bigskip{\centering\Large Travel \hfill \$40,000}\medskip

	\noindent\textit{Justification}:
	Engagement in national and international conferences is indispensable for disseminating our research findings and establishing meaningful relationships with peer researchers in both theoretical and applied topology. To facilitate this, our budget earmarks funds for travel, accommodation, and registration fees, thereby enabling each of our team members to participate in at least two significant conferences annually. One of these is specifically ATMCS (Algebraic Topology: Methods, Computation, and Science), a premier event in the field of computational topology. We've designated \$8,000 per annum for this purpose, breaking it down to a \$2,000 allowance per team member (based on a team of four annually). This calculation takes into account an estimated \$1,000 for international airfare, \$400 for domestic travel, and \$600 allocated for lodging and daily expenses.

	\bigskip{\centering\Large Dissemination Costs \hfill \$10,000}\medskip

	\noindent\textit{Justification}:
	The dissemination of our research findings in both applied and theoretical topology calls for publication in high-impact, peer-reviewed journals. These often entail associated costs, such as open-access fees. To accommodate this, I have allocated a yearly budget of \$2,000 specifically for publication fees, enabling us to target top-tier journals and thereby amplify the reach and impact of our work. This allocation is grounded in historical expenditure patterns and is indispensable for fulfilling the project's objectives, aligning closely with the NSERC Discovery Grant's emphasis on high-quality research and effective dissemination.
\end{document}