\documentclass[12pt]{article}
\usepackage{amsmath}
\usepackage{microtype}
\usepackage{amssymb}
\usepackage{mathtools}
\usepackage{tikz-cd}
\usepackage{mathbbol} % changes \mathbb{} and adds more support
\usepackage{csquotes}
\usepackage{enumitem}
\usepackage{setspace} % For line spacing
\usepackage{mathptmx} % Times New Roman for both text and math
\usepackage[scaled=.92]{helvet} % Helvetica, scaled to match Times New Roman
\usepackage{layout}

% set up 12pt Times New Roman
\renewcommand{\rmdefault}{ptm}
\renewcommand{\sfdefault}{phv}
\renewcommand{\ttdefault}{pcr}

%\usepackage{layout}
\pagestyle{empty}

\setlength{\parindent}{0pt}
\usepackage[letterpaper,
	margin=1.87cm,
	hoffset=0cm,
	voffset=0cm,
	headheight=0cm,
	headsep=0cm,
	marginparwidth=0cm,
	marginparsep=0cm,
	footskip=0cm]{geometry} % Sets paper size and margins

% line spacing
%\onehalfspacing % Alternatively, you can use \singlespacing to try to meet the "no more than six lines per inch" requirement

% bibliography
\usepackage[
	backend=biber,
	style=numeric, % alphabetic, numeric, remove sorting=none
	sorting=nyt,
	backref=false,
	url=false,
	doi=false,
	isbn=false,
	eprint=false]{biblatex}

\setlength{\bibitemsep}{0.0cm}
\renewbibmacro{in:}{}  % don't display "in:" before the journal name
\AtEveryBibitem{\clearfield{pages}}  % don't show page numbers

% Customize subsection title format
\usepackage{titlesec}
\titleformat{\subsection}[runin]
{\normalfont\bfseries}
{\thesubsection.}
{1em}
{}
[:] % Append a colon at the end

\begin{document}
	\textsc{\large Effective Algebro-Homotopical Constructions with Applications to Data Science and Quantum Field Theory}
	--- \hfill \textsc{\large Anibal M. Medina-Mardones}

	\begin{center}
		\Huge Budget Justification
	\end{center}

	{\centering\Large Total \hfill \$414,000}\bigskip
%	\textbf{Grand Total:} \$444,000
	\par This budget has been meticulously crafted to align with the research objectives outlined in our proposal and is critical for the project's successful completion. We will manage the funds judiciously to maximize both the impact of our research and the preparation of Highly Qualified Personnel in the domains of applied and theoretical topology.
	All values are presented in CAD.

	\bigskip{\centering\Large Personnel \hfill \$330,000}\bigskip

	\bigskip{\centering\large MSc Students (4) \hfill \$60,000}\bigskip

	\noindent\textit{Justification}:
	The MSc students will be responsible for most of the software development of Objectives (1) and (2) of our proposal. They will also contribute to writing research papers and will be co-authors on at least one paper during the grant period. The budget will cover their stipends, in alignment with Western's Mathematics Department guidelines for graduate student compensation (\$7,500 per student per year).
	We mention that, although the program requires 6 MSc students, we request funds for 4 since two are already covered using the starting grant of the PI.

	\bigskip{\centering\large PhD Students (4) \hfill \$150,000}\bigskip

	\noindent\textit{Justification}:
	PhD students will take on more specialized roles in the project, focusing on deeper challenges in Objectives (1) and (2) of our proposal. Their role will be to provide solutions to hard mathematical problems and guide MSc students through basic topological and algorithmic concepts. They will also be responsible for developing and implementing algorithms. Their roles are crucial for the project's long-term goals, including publications in high-impact journals. The budget will cover their stipends, in alignment with university guidelines for graduate student compensation.

	\bigskip{\centering\large Postdoctoral Researcher (1) \hfill \$120,000}\bigskip

	\noindent\textit{Justification}:
	The Postdoctoral Researcher will focus on the deepest of the investigations of our project, Objective (3), which will bridge the state sum and functorial classification of topological phases. This part of the budget will cover part of their salary and benefits in line with institutional and national standards. The remainder part will be covered by the Perimeter Institute, where the Postdoctoral Researcher will have a dual appointment for 3 years.

	\bigskip{\centering\Large Travel and Conferences \hfill \$40,000}\bigskip

	\noindent\textit{Justification}:
	Attending national and international conferences is crucial for disseminating our research findings and for networking with other researchers in the field of theoretical and applied topology. This budget will cover travel, accommodation, and registration fees for the project team to attend at least two conferences per year.
	The most important venue for Topological Data Analysis is the yearly conference

	\newpage
	\bigskip{\centering\Large Equipment \hfill \$24,000}\bigskip

	\noindent\textit{Justification}: Considering that a significant portion of our work entails software development and data analysis, it is imperative to equip each Highly Qualified Personnel (HQP) in our program with the necessary computational tools.
	Accordingly, we have allocated \$2,000 per team member for the acquisition of a laptop.
	This accounts for the entirety of our equipment-related expenses.

	\bigskip{\centering\Large Dissemination Total \hfill \$10,000}\bigskip

	\noindent\textit{Justification}:
	The dissemination of our research findings in applied and theoretical topology necessitates publication in high-impact, peer-reviewed journals, which often require associated fees such as open-access charges. A yearly allocation of \$2,000 for publication fees allows us to target top-tier journals, enhancing both the reach and impact of our research. This budget is based on historical costs and is essential for achieving the project’s objectives in line with the NSERC Discovery Grant's focus on high-quality research and dissemination.

	\bigskip{\centering\Large Dissemination Total \hfill \$10,000}\bigskip

	\noindent\textit{Justification}:
	This category is allocated to accommodate unforeseen expenditures that may arise during the course of the project. It is designed to ensure the project's smooth operation by covering unexpected costs. These could include, but are not limited to, hardware replacements, software upgrades, and additional data storage needs. Moreover, this allocation could cover emergency costs related to data recovery or specialized technical services, as well as unexpected publication fees. Having a budget for miscellaneous expenses increases the project's adaptability to changing conditions and helps mitigate risks, ensuring that the project timeline and objectives are not compromised.
\end{document}