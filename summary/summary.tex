\documentclass[12pt]{article}
\usepackage{amsmath}
\usepackage{microtype}
\usepackage{amssymb}
\usepackage{mathtools}
\usepackage{tikz-cd}
\usepackage{mathbbol} % changes \mathbb{} and adds more support
\usepackage{csquotes}
\usepackage{enumitem}
\usepackage{setspace} % For line spacing
\usepackage{mathptmx} % Times New Roman for both text and math
\usepackage[scaled=.92]{helvet} % Helvetica, scaled to match Times New Roman
\usepackage{layout}

% set up 12pt Times New Roman
\renewcommand{\rmdefault}{ptm}
\renewcommand{\sfdefault}{phv}
\renewcommand{\ttdefault}{pcr}

%\usepackage{layout}
\pagestyle{empty}

\setlength{\parindent}{0pt}
\usepackage[letterpaper,
	margin=1.87cm,
	hoffset=0cm,
	voffset=0cm,
	headheight=0cm,
	headsep=0cm,
	marginparwidth=0cm,
	marginparsep=0cm,
	footskip=0cm]{geometry} % Sets paper size and margins

% line spacing
%\onehalfspacing % Alternatively, you can use \singlespacing to try to meet the "no more than six lines per inch" requirement

% bibliography
\usepackage[
	backend=biber,
	style=numeric, % alphabetic, numeric, remove sorting=none
	sorting=nyt,
	backref=false,
	url=false,
	doi=false,
	isbn=false,
	eprint=false]{biblatex}

\setlength{\bibitemsep}{0.0cm}
\renewbibmacro{in:}{}  % don't display "in:" before the journal name
\AtEveryBibitem{\clearfield{pages}}  % don't show page numbers
% environments
\newtheorem{theorem}{Theorem}
\newtheorem*{theorem*}{Theorem}
\newtheorem{proposition}[theorem]{Proposition}
\newtheorem*{proposition*}{Proposition}
\newtheorem{lemma}[theorem]{Lemma}
\newtheorem*{lemma*}{Lemma}
\newtheorem{corollary}[theorem]{Corollary}
\newtheorem*{corollary*}{Corollary}

\theoremstyle{definition}
\newtheorem{definition}[theorem]{Definition}
\newtheorem*{definition*}{Definition}
\newtheorem{remark}[theorem]{Remark}
\newtheorem*{remark*}{Remark}
\newtheorem{example}[theorem]{Example}
\newtheorem*{example*}{Example}
\newtheorem{construction}[theorem]{Construction}
\newtheorem*{construction*}{Construction}
\newtheorem{convention}[theorem]{Convention}
\newtheorem*{convention*}{Convention}
\newtheorem{terminology}[theorem]{Terminology}
\newtheorem*{terminology*}{Terminology}
\newtheorem{notation}[theorem]{Notation}
\newtheorem*{notation*}{Notation}
\newtheorem{question}[theorem]{Question}
\newtheorem*{question*}{Question}

% hyphenation
\hyphenation{co-chain}
\hyphenation{co-chains}
\hyphenation{co-al-ge-bra}
\hyphenation{co-al-ge-bras}
\hyphenation{co-bound-ary}
\hyphenation{co-bound-aries}
\hyphenation{Func-to-rial-i-ty}
\hyphenation{colim-it}
\hyphenation{di-men-sional}

% basics
\DeclareMathOperator{\face}{d}
\DeclareMathOperator{\dege}{s}
\DeclareMathOperator{\bd}{\partial}
\DeclareMathOperator{\sign}{sign}
\newcommand{\ot}{\otimes}
\DeclareMathOperator{\EZ}{EZ}
\DeclareMathOperator{\AW}{AW}

% sets and spaces
\newcommand{\N}{\mathbb{N}}
\newcommand{\Z}{\mathbb{Z}}
\newcommand{\Q}{\mathbb{Q}}
\newcommand{\R}{\mathbb{R}}
\renewcommand{\k}{\Bbbk}
\newcommand{\sym}{\mathbb{S}}
\newcommand{\cyc}{\mathbb{C}}
\newcommand{\Ftwo}{{\mathbb{F}_2}}
\newcommand{\Fp}{{\mathbb{F}_p}}
\newcommand{\Cp}{{\cyc_p}}
\newcommand{\gsimplex}{\mathbb{\Delta}}
\newcommand{\gcube}{\mathbb{I}}

% categories
\newcommand{\Cat}{\mathsf{Cat}}
\newcommand{\Fun}{\mathsf{Fun}}
\newcommand{\Set}{\mathsf{Set}}
\newcommand{\Top}{\mathsf{Top}}
\newcommand{\CW}{\mathsf{CW}}
\newcommand{\Ch}{\mathsf{Ch}}
\newcommand{\simplex}{\triangle}
\newcommand{\sSet}{\mathsf{sSet}}
\newcommand{\cube}{\square}
\newcommand{\cSet}{\mathsf{cSet}}
\newcommand{\Alg}{\mathsf{Alg}}
\newcommand{\coAlg}{\mathsf{coAlg}}
\newcommand{\biAlg}{\mathsf{biAlg}}
\newcommand{\sGrp}{\mathsf{sGrp}}
\newcommand{\Mon}{\mathsf{Mon}}
\newcommand{\symMod}{\mathsf{Mod}_{\sym}}
\newcommand{\symBimod}{\mathsf{biMod}_{\sym}}
\newcommand{\operads}{\mathsf{Oper}}
\newcommand{\props}{\mathsf{Prop}}

% operators
\DeclareMathOperator{\free}{F}
\DeclareMathOperator{\forget}{U}
\DeclareMathOperator{\yoneda}{\mathcal{Y}}
\DeclareMathOperator{\Sing}{Sing}
\newcommand{\loops}{\Omega}
\DeclareMathOperator{\cobar}{\mathbf{\Omega}}
\DeclareMathOperator{\proj}{\pi}
\DeclareMathOperator{\incl}{\iota}
\DeclareMathOperator{\Sq}{Sq}
\DeclareMathOperator{\ind}{ind}

% chains
\DeclareMathOperator{\chains}{N}
\DeclareMathOperator{\cochains}{N^{\vee}}
\DeclareMathOperator{\gchains}{C}

% pair delimiters (mathtools)
\DeclarePairedDelimiter\bars{\lvert}{\rvert}
\DeclarePairedDelimiter\norm{\lVert}{\rVert}
\DeclarePairedDelimiter\angles{\langle}{\rangle}
\DeclarePairedDelimiter\set{\{}{\}}
\DeclarePairedDelimiter\ceil{\lceil}{\rceil}
\DeclarePairedDelimiter\floor{\lfloor}{\rfloor}

% other
\newcommand{\id}{\mathsf{id}}
\renewcommand{\th}{\mathrm{th}}
\newcommand{\op}{\mathrm{op}}
\DeclareMathOperator*{\colim}{colim}
\DeclareMathOperator{\coker}{coker}
\newcommand{\Hom}{\mathrm{Hom}}
\newcommand{\End}{\mathrm{End}}
\newcommand{\coEnd}{\mathrm{coEnd}}
\newcommand{\xla}[1]{\xleftarrow{#1}}
\newcommand{\xra}[1]{\xrightarrow{#1}}
\newcommand{\defeq}{\stackrel{\mathrm{def}}{=}}

% comments
\newcommand{\anibal}[1]{\noindent\textcolor{blue}{\underline{Anibal}: #1}}

% pdf
\newcommand{\pdfEinfty}{\texorpdfstring{${E_\infty}$}{E-infty}}

% mathrm
\newcommand{\rA}{\mathrm{A}}
\newcommand{\rB}{\mathrm{B}}
\newcommand{\rC}{\mathrm{C}}
\newcommand{\rD}{\mathrm{D}}
\newcommand{\rE}{\mathrm{E}}
\newcommand{\rF}{\mathrm{F}}
\newcommand{\rG}{\mathrm{G}}
\newcommand{\rH}{\mathrm{H}}
\newcommand{\rI}{\mathrm{I}}
\newcommand{\rJ}{\mathrm{J}}
\newcommand{\rK}{\mathrm{K}}
\newcommand{\rL}{\mathrm{L}}
\newcommand{\rM}{\mathrm{M}}
\newcommand{\rN}{\mathrm{N}}
\newcommand{\rO}{\mathrm{O}}
\newcommand{\rP}{\mathrm{P}}
\newcommand{\rQ}{\mathrm{Q}}
\newcommand{\rR}{\mathrm{R}}
\newcommand{\rS}{\mathrm{S}}
\newcommand{\rT}{\mathrm{T}}
\newcommand{\rU}{\mathrm{U}}
\newcommand{\rV}{\mathrm{V}}
\newcommand{\rW}{\mathrm{W}}
\newcommand{\rX}{\mathrm{X}}
\newcommand{\rY}{\mathrm{Y}}
\newcommand{\rZ}{\mathrm{Z}}
% mathcal
\newcommand{\cA}{\mathcal{A}}
\newcommand{\cB}{\mathcal{B}}
\newcommand{\cC}{\mathcal{C}}
\newcommand{\cD}{\mathcal{D}}
\newcommand{\cE}{\mathcal{E}}
\newcommand{\cF}{\mathcal{F}}
\newcommand{\cG}{\mathcal{G}}
\newcommand{\cH}{\mathcal{H}}
\newcommand{\cI}{\mathcal{I}}
\newcommand{\cJ}{\mathcal{J}}
\newcommand{\cK}{\mathcal{K}}
\newcommand{\cL}{\mathcal{L}}
\newcommand{\cM}{\mathcal{M}}
\newcommand{\cN}{\mathcal{N}}
\newcommand{\cO}{\mathcal{O}}
\newcommand{\cP}{\mathcal{P}}
\newcommand{\cQ}{\mathcal{Q}}
\newcommand{\cR}{\mathcal{R}}
\newcommand{\cS}{\mathcal{S}}
\newcommand{\cT}{\mathcal{T}}
\newcommand{\cU}{\mathcal{U}}
\newcommand{\cV}{\mathcal{V}}
\newcommand{\cW}{\mathcal{W}}
\newcommand{\cX}{\mathcal{X}}
\newcommand{\cY}{\mathcal{Y}}
\newcommand{\cZ}{\mathcal{Z}}
% mathsf
\newcommand{\sA}{\mathsf{A}}
\newcommand{\sB}{\mathsf{B}}
\newcommand{\sC}{\mathsf{C}}
\newcommand{\sD}{\mathsf{D}}
\newcommand{\sE}{\mathsf{E}}
\newcommand{\sF}{\mathsf{F}}
\newcommand{\sG}{\mathsf{G}}
\newcommand{\sH}{\mathsf{H}}
\newcommand{\sI}{\mathsf{I}}
\newcommand{\sJ}{\mathsf{J}}
\newcommand{\sK}{\mathsf{K}}
\newcommand{\sL}{\mathsf{L}}
\newcommand{\sM}{\mathsf{M}}
\newcommand{\sN}{\mathsf{N}}
\newcommand{\sO}{\mathsf{O}}
\newcommand{\sP}{\mathsf{P}}
\newcommand{\sQ}{\mathsf{Q}}
\newcommand{\sR}{\mathsf{R}}
\newcommand{\sS}{\mathsf{S}}
\newcommand{\sT}{\mathsf{T}}
\newcommand{\sU}{\mathsf{U}}
\newcommand{\sV}{\mathsf{V}}
\newcommand{\sW}{\mathsf{W}}
\newcommand{\sX}{\mathsf{X}}
\newcommand{\sY}{\mathsf{Y}}
\newcommand{\sZ}{\mathsf{Z}}
% mathbb
\newcommand{\bA}{\mathbb{A}}
\newcommand{\bB}{\mathbb{B}}
\newcommand{\bC}{\mathbb{C}}
\newcommand{\bD}{\mathbb{D}}
\newcommand{\bE}{\mathbb{E}}
\newcommand{\bF}{\mathbb{F}}
\newcommand{\bG}{\mathbb{G}}
\newcommand{\bH}{\mathbb{H}}
\newcommand{\bI}{\mathbb{I}}
\newcommand{\bJ}{\mathbb{J}}
\newcommand{\bK}{\mathbb{K}}
\newcommand{\bL}{\mathbb{L}}
\newcommand{\bM}{\mathbb{M}}
\newcommand{\bN}{\mathbb{N}}
\newcommand{\bO}{\mathbb{O}}
\newcommand{\bP}{\mathbb{P}}
\newcommand{\bQ}{\mathbb{Q}}
\newcommand{\bR}{\mathbb{R}}
\newcommand{\bS}{\mathbb{S}}
\newcommand{\bT}{\mathbb{T}}
\newcommand{\bU}{\mathbb{U}}
\newcommand{\bV}{\mathbb{V}}
\newcommand{\bW}{\mathbb{W}}
\newcommand{\bX}{\mathbb{X}}
\newcommand{\bY}{\mathbb{Y}}
\newcommand{\bZ}{\mathbb{Z}}
% mathfrak
\newcommand{\fA}{\mathfrak{A}}
\newcommand{\fB}{\mathfrak{B}}
\newcommand{\fC}{\mathfrak{C}}
\newcommand{\fD}{\mathfrak{D}}
\newcommand{\fE}{\mathfrak{E}}
\newcommand{\fF}{\mathfrak{F}}
\newcommand{\fG}{\mathfrak{G}}
\newcommand{\fH}{\mathfrak{H}}
\newcommand{\fI}{\mathfrak{I}}
\newcommand{\fJ}{\mathfrak{J}}
\newcommand{\fK}{\mathfrak{K}}
\newcommand{\fL}{\mathfrak{L}}
\newcommand{\fM}{\mathfrak{M}}
\newcommand{\fN}{\mathfrak{N}}
\newcommand{\fO}{\mathfrak{O}}
\newcommand{\fP}{\mathfrak{P}}
\newcommand{\fQ}{\mathfrak{Q}}
\newcommand{\fR}{\mathfrak{R}}
\newcommand{\fS}{\mathfrak{S}}
\newcommand{\fT}{\mathfrak{T}}
\newcommand{\fU}{\mathfrak{U}}
\newcommand{\fV}{\mathfrak{V}}
\newcommand{\fW}{\mathfrak{W}}
\newcommand{\fX}{\mathfrak{X}}
\newcommand{\fY}{\mathfrak{Y}}
\newcommand{\fZ}{\mathfrak{Z}}

\addbibresource{../usualpapers.bib}
\addbibresource{../bibliography.bib}

\begin{document}
	\textsc{\large Effective Algebro-Homotopical Constructions with Applications to Data Science and Quantum Field Theory}
	--- \hfill \textsc{\large Anibal M. Medina-Mardones}
	\hrule

	\bigskip
	\begin{center}
		\Large{Summary}
	\end{center}

	\textbf{Introduction}

	\smallskip Since its inception, the field of algebraic topology (AT) has grappled with a fundamental trade-off between constructiveness and functoriality. For the better part of the last five decades, non-constructive approaches have taken centre stage, producing numerous powerful and elegant results. However, the growing applicability of AT in scientific and computational fields has created the need for more constructive approaches to these developments.

	\smallskip\textbf{Long-term vision}

	\smallskip I aim to provide effective constructions of sophisticated homotopical invariants and, by doing so, allow for the application of some of the profound ideas in AT to fields beyond pure mathematics, including data science and quantum field theory.

	\smallskip\textbf{Program's objectives}

	\smallskip
	(1) Compute primary \& secondary cohomology operations of triangulated spaces. \hfill (MSc1-2, PhD1-2)\par
	(2) Compute Steenrod operations on Khovanov homology of knots and links. \hfill (MSc3, PhD3)\par
	(3) Compute Steenrod barcodes and persistent cup-lengths of real-world datasets. \hfill (MSc4-5, PhD4)\par
	(4) Build a bridge between state sum and functorial topological quantum field theories. \hfill (MSc6, PostDoc)\par

	\smallskip\textbf{Key ideas}

	\smallskip Obj (1) rests on the constructive description, obtained by the author and other researchers, of a complete algebro-homotopical invariant of triangulated spaces, a so-called E-infinity algebra structure on their cochains. The fact that this invariant captures all homotopical information is a jewel of modern homotopy theory.

	\smallskip Obj (2) hinges on a recent result by a couple of teams showing that Khovanov homology, a powerful invariant for knots and links, is derived from a spectrum in the sense of stable homotopy theory. This spectrum can be modelled using an object akin to a triangulated space, enabling the adaptation of our effective construction from Obj (1).

	\smallskip Obj (3) is based on the adaptation to nested families of the effective constructions of cohomology operations for triangulated spaces of Obj (1). This is because persistent cohomology, a cornerstone of Topological Data Analysis, is based on the study of such nested families.

	\smallskip Obj (4) relies on the free n-category generated by the n-simplex. This object, previously related by the author to cohomology operations, belongs to both; the lattice-theoretic domain of triangulated space-times and the higher categorical realm of the cobordism hypothesis.

	\smallskip\textbf{Deliverables}

	\smallskip 1) At least 20 publications in high-impact journals disseminating our work.

	\smallskip 2) Modules in the computer algebra systems \texttt{SAGE} or \texttt{Maple} for the computation of operations on the cohomology of triangulated spaces and Khovanov homology of knots.

	\smallskip 3) High-performance routines in \texttt{giottoTDA} for the computation of Steenrod barcodes and persistent cup-length.

	\smallskip\textbf{Mentorship}

	\smallskip With its dual emphasis on pure and applied mathematics, our program will prepare 10 trainees for diverse careers, whether they are exploring the abstract landscapes of academic research or the concrete challenges present in industry and public service.
\end{document}

Proposal

Five pages max.

Addressing the points below, describe the proposed research to be supported. Images and graphics are included in the page limit.

Recent progress

Describe your recent progress in research activities related to the proposal; for returning grantees, describe as well the progress attributable to your previous Discovery Grant.

Objectives

Define the short- and long-term objectives of your research program. Note that a research program should have a long-term vision that expands beyond the five years of the Discovery Grant. A single, short-term project or collection of projects does not constitute a research program.

Literature review

Discuss the literature pertinent to the proposal, placing the proposed research in the context of the state of the art.

Methodology

Describe the methods and proposed approach, providing sufficient details to allow the reviewers to assess the feasibility of the research activities.

Considering equity, diversity and inclusion (EDI) in the research process promotes research excellence by making research outcomes more ethically sound, rigorous, reproducible, and useful. It is important to consider EDI through each stage of the research process including, but not limited to, the research questions, design, methodology, analysis, interpretation and dissemination of results, and integrate these considerations where relevant. Consult Equity, diversity and inclusion considerations at each stage of the research process for more information.

Impact

Explain the anticipated significance of the work.

Note:

If the information provided is insufficient, NSERC reserves the right to take appropriate action, such as not soliciting reports from external reviewers or withdrawing the application from the competition.
If relevant to your research, consult NSERC’s Guidelines for the preparation and review of applications in interdisciplinary research and/or Guidelines for the preparation and review of applications in engineering and the applied sciences.
If NSERC determines that the subject matter is outside of its mandate at any time during the review cycle, the application will be rejected. See This link will take you to another Web site Selecting the appropriate federal granting agency for more information.
